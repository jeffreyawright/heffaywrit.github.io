\chapter{BREAKING BLUE: RACE, ETHNICITY \& PARTISAN CHANGE IN THE STATE OF GEORGIA} \label{racechapter}

 \textit{This chapter focuses on the issue of race, ethnicity, and partisan realignment, exploring the hypothesis that recent electoral cycles in the state of Georgia were characterized by defection from the Republican ticket by White voters at levels that sharply accelerated secular trends. The expectation is that this deviation largely explains Democratic candidate Joe Biden's victory over Republican candidate Donald Trump in Georgia's 2020 presidential elections, one of two long-standing Republican strongholds. It is argued that the conditions exist to define 2016 and 2020 as realignment elections, signaled by the state-level partisan change of White Republicans.}

\doublespacing

\section{Introduction}

The surprise victory of Democratic challenger Joe Biden over incumbent Donald Trump in the Republican stronghold of Georgia generated debate and speculation over the role that race \& ethnicity played in the outcome. Some attributed the upset to the increasing proportion of non-White voters in Georgia, particularly in Atlanta's suburbs; others proclaimed it the fruit of years grassroots minority-voter campaigns and their tireless organizers, embodied by Democratic leader Stacey Abrams; and still others speculated that the polarizing figure of Trump had alienated White suburban females, and perhaps no small number of male voters \citep{bailey_stacey_2020, epstein_10-year_2021, cheney-rice_georgia_2020, cohn_why_2021, herndon_future_2020, perry_how_2020, somvichian-clausen_how_2020}. Adjudicating the importance of demography, voter mobilization, and partisan change to the electoral upset is important both as matter of political demography and political science. 

The research problem is framed by the hypothesis put forth 20 years ago by \citet{judis_emerging_2002} in \textit{The Emerging Democratic Majority}. Their essential argument garnered widespread of media attention: A youthful and growing population of non-White voters, coupled with the rapidly aging population structure of White voters, would propel the Democratic party to decisive partisan dominance. Their assumptions are in fact more complex than a racial headcount. But much like the prescient, nuanced observations put forth by \citeauthor{phillips_emerging_2015} (1969) in \textit{The Emerging Republican Majority} have long been reduced to a Southern Strategy trope, the lay understanding of Judis and Teixeira's work is encapsulated in the summary term `Demography is Destiny.' 

However, Judis and Teixeira's thesis envisioned a coalition of `progressive centrism' uniting minority voters with segments of the White electorate. They expected the Democratic party to retain roughly half of the White working class while gaining an increasing advantage with White women and professionals alienated by the Republican party's hardening positions on cultural issues and traditional values. In 2008, following the election of Barack Obama, Judis wrote of the Democratic majority, "it emerged!" \citep{judis_america_2008}. In 2015, noting that support of White working class voters for a new Democratic majority had collapsed, he reversed himself to declare it `a mirage' in a piece entitled \textit{The Emerging Republican Advantage} \citep{judis_emerging_2015}. Then, months before the 2020 election Judis disavowed that thesis, noting that White voters were defecting from Trump's Republican party, which he expected might hand Biden a victory. Significantly, he said he no longer regarded minority voters as a `magic bullet' for the Democratic party, predicting that an increasing number of voters with minority parents will be identifying as White \citep{judis_talking_2020}.   

Such equivocation from a leading voice on the matter illustrates that the debate on race and partisan realignment in the U.S. remains far from settled. Georgia, a state that underwent an intensely researched partisan realignment from solidly Democratic prior to the U.S. Civil Rights movement to Republican trifecta today, presents an opportune case to advance this scholarship. This study seeks to determine the role of race/ethnicity in the 2016 and 2020 Georgia elections, to both quantify the composition of the constituency who awarded Biden the state's electoral votes victory, and to assess the relevance of partisan change to the outcome. The analysis is informed by critical election theory, hewing roughly to the definition of, ``[t]he election in which the change in the parties' balance of power is first evident'' \citep[pg. 362]{campbell_party_2006}

\section{Literature Review}

The \textit{Emerging Democratic Majority} theory is rooted in the two-party electoral system in the United States, and the idea that under the right circumstances one major party can hold greater sway over the electorate than the other. These conditions are also assumed to be to some extent predictable. Broadly speaking, the aggregate support for each party can be estimated based on the number of eligible voters, the distribution of partisanship among them, and the likelihood that partisans will vote. Such napkin calculations can be made at various geographic levels, as well as projected into the future. Each stage of this stylized equation is the subject of a vast body of scholarship. The variables affecting voter turnout is one of the most extensively studied subject of American politics: including, but not limited to, race/ethnicity, income, marital status, education, sex, and age \citep{verba_voice_1995, wolfinger_who_1980, niemi_controversies_1984}. The distribution of partisanship across such sociodemographic variables is largely predictable due to their strong association with the long-term, stable, psychological nature of partisan identification \citep{campbell_american_1960, converse_time_1969, green_partisan_2008}. The past, current, and future demographic composition of the electorate can be estimated or projected my measuring the demographic components of change, fertility, mortality, and migration, which, like voter turnout and partisan preferences, vary according to race/ethnicity, SES, age cohort, and so on \citep{shryock_methods_1975, rowland_demographic_2003, smith_state_2006}. Taken together, changes in turnout, partisan preference, and demographic composition can create shifts in the balance of power. 

By definition, realignment requires an enduring and consequential change in the macropartisanship of the electorate \citep{meffert_realignment_2001}. Macropartisanship reflects the aggregate distribution of partisan identification in the electorate; micropartisanship, in contrast, defines individual party identification that underpin it \citep{andersen_after_1996}. Given the two-party system of competition that characterizes U.S. democracy, macropartisanship was proposed to center around a median voter equilibrium \citep{downs_economic_1957}. The equilibrium is not necessarily even. In defined periods, one party has tended to enjoy an electoral advantage (or at least hold the White House). It has often been the case that if one party accumulates substantial competitive advantage, partisanship shifts in response. Scholars generally recognize five distinct eras, or Party Systems, in U.S. political history, which are characterized by a period of relative equilibrium punctuated by realignment: Democratic-Republicans and Federalists (1796-1816); Democrats and Whigs (1840-1856); Republicans and Democrats (1860-1896); Republicans and Democrats (1896–1932); Democrats and Republicans (1932- ). The endpoints of each Party System are generally identified with a significant election that altered the balance of power, such as Andrew Jackson splitting the Democratic-Republicans after the Federalists had withered and winning the 1828 election, or the New Deal coalition forged by Franklin Roosevelt in 1932, leading to a long stretch of Democratic party dominance after the previous, Republican era collapsed with the Great Recession. 

The realignment literature initially focused on the mobilization and conversion theses as sources of realignment \citep{campbell_american_1960}. Mobilization describes a shift produced when partisan preferences remain stable but the composition of the electorate changes; conversion is produced by the movement of committed partisans from one party to the opposing party \citep{andersen_after_1996, darmofal_dynamics_2010}. Mobilization refers to the activation of nonvoters into the electorate by political campaigns, grassroots organization, issue interest, or other factors. Nonvoters include those who have never voted and those who have but generally forego participation. If a particularly salient election activated a large number of nonvoters for a single election only, it would not translate into realignment, however that is significantly unlikely given that voting is habitual. Additionally, voters with a different partisan profile than the status quo may be incorporated into the electorate by changes to eligibility/disenfranchisement laws, the incorporate of a territory as a state, and so forth. Demobilization, while generally considered separately as part of the dealignment literature, is effectively the inverse. Partisans abstain from voting at different rates by party ID because of an unpopular candidate, an alienating campaign, a withdrawal of campaigning resources, a lack of highly controversial issues, failure to field a candidate, or similar factors. Although the literature was preoccupied with dealignment in terms of declining voter participation and a rise of independent partisans, \citet{mcdonald_myth_2001} demonstrated that declining turnout rates was largely a statistical artifice, while numerous empirical studies have shown that most independents behave as partisans, and the share of truly independent swing voters is generally stable \citep{keith_partisan_1986}. However, while dealignment may not have changed macropartisanship, lower participation and third-party voting are often present in realignment elections  \citep{stanley_partisan_1988, hammond_minor_1976}.

Researchers later established the importance of generational change as well \citep{miller_generational_1992, black_rise_2009, carmines_issue_1981}. From the perspective of demography, change to the partisan profile of the population could take many forms. Conservative groups such as evangelicals and Mormons have higher fertility rates than secular families, and most minority groups in turn have higher fertility rates than Whites families. Mortality varies among groups as well. It is estimated, for example, that differential death rates between White and Black U.S. citizens since 1970 led to 1.7 million fewer Black voters in the 2004 elections \citep{rodriguez_black_2015}. Immigration has changed the race/ethnic composition of the U.S. electorate both through naturalization and jus solis citizenship of native-born descendants. Interstate migration of citizens could potentially alter the distribution of Electoral College votes. The literature tends to engage generational change under the rubric of mobilization, as it does not involve micropartisan change, though in this study it is called the `cohort progression' effect to recognize the separate importance of demographic change to the future balance of partisan power in the U.S.     

Conversion drives realignment when micropartisan change is present. Naturally, conversion can be accompanied by other effects. For example, \citet{campbell_sources_1985} finds that one-third of the New Deal realignment stemmed from individual voters switching parties, while two-thirds could be attributed to changes in the electorate. If defection characterizes a single election before partisan voting regresses to the mean, it would not qualify as partisan realignment, but rather as swing voting. True swing voters represent a small and declining share of the electorate \citep{smidt_polarization_2017}. Candidates, issues, scandal, and other temporal events may lead to vote swings. On the other hand, when widespread change in micropartisanship alters macropartisanship over a longer time frame, conversion is present. Disentangling conversion from cohort progression and mobilization is fraught with peril. Election returns are calculated as counts but masked by voter secrecy; in some cases partisan ID is proxied by partisan primary participation or registration, but in general partisan ID is estimated by surveys, and subject to error and reporting biases.  

\subsection{Critical Elections}

V.O. Key characterized critical elections as those with usher in a `sharp and durable' upheaval in the central tendency of partisan power, later dubbed realignment elections in the seminal work The American Voter \citep{key_jr_theory_1955, campbell_american_1960}. Key also outlined an alternative, gradual process of secular partisan change that registers incrementally in successive elections over time \citep{key_jr_secular_1959}. The conditions favoring critical partisan breaks include ideological polarization, an increasing salience of third party candidates, issue-based campaigning and voting, and the growth of independent partisanship. Various theorists have refined the concept, characterizing surge, maintenance, and deviating elections; identifying accompanying conditions such as political polarization, social unrest, and third-party challengers; and recognizing that critical elections may involve more than a single election contest and may be geographically uneven among subnational states or regions \citep{sellers_equilibrium_1965, pomper_classification_1967, meffert_realignment_2001, campbell_american_1960, lichtman_critical_1976}. 

The subject of the Southern realignment generated an enormous volume of research but little consensus on the question of critical realignment \citep{buchanan_realignment_2002, beck_partisan_1977, hammond_minor_1976, bartels_partisanship_2000, black_rise_2009, black_transformation_2004, campbell_party_2006, schickler2016racial}. In the aftermath of the 1964 landslide victory of Democrat Lyndon Johnson, academics began a decades-long debate as to whether it qualified as a critical election, one marking the end of the New Deal era. Johnson had signed the Civil Rights Act in April of that year, remarking to special assistant Bill Moyers, ``I think we just delivered the South to the Republican party for a long time to come'' \citep{moyers_moyers_2015}. In the aftermath, the Republican party nominated Barry Goldwater, who opposed the bill, to run against Johnson, and Democratic Senator Strom Thurmond of South Carolina, who headed the segregationist Dixiecrat party's presidential bid in 1948, renounced his party and joined the GOP. In the Solid South, where the Democratic party functioned as a one-party White supremacist regime, Black voters overwhelmingly defected to the Johnson ticket while White voters delivered to Goldwater five states --- the only states he won outside of Arizona  \citep{converse_change_1972, segal_partisan_1968}. The debate was partly fueled by the 1967 launch of segregationist George Wallace's third-party campaign, which went on to sweep the Deep South in the 1968 presidential contest, and again when Nixon crushed Democratic opponent George McGovern in 1972, regaining the White House for the GOP. To this day the debate remains unresolved and the timing of the end of the fifth Party System and the beginning of the sixth unsettled --- with the additional wrinkle that while the Democratic party has generally been the dominant party in terms of party identification, it has not enjoyed a corresponding dominance of the White House or Congress \citep{lawrence_collapse_1997, mayer_divided_1996}. Some scholars have portrayed post-Solid South partisanship as characterized by dealignment, the rise of independent partisans \citep{carmines_unrealized_1987, beck_partisan_1977}.

Partisan realignment looms large in importance to the study of U.S. electoral politics. Partisanship is one of three determinants of electoral outcomes, along with turnout and composition of the electorate. Partisan ID is a stable characteristic and increasingly important as independent `floater voters' who switch parties from one election to the next have steadily declined to between 5 and 7 percent of votes cast in recent contests \citep{smidt_polarization_2017}. Political science has long grappled with the relationship of partisanship to partisan change \citep{green_how_1994, jackson_connecting_2011}. Connecting tectonic macropartisan shifts of electoral events that are measured in ballot counts to the complex social psychology of micro-level partisan identification is problematic both conceptually and methodologically.  The fundamental debate on individual-level partisanship divides scholars who conceive of it as a function of rational evaluations by individuals who keep a `running tally' of parties and with the party whose policies are most beneficial \citep{downs_economic_1957, fiorina_retrospective_1981}, and those who regard it as a function of group membership \citep{campbell_american_1960, green_partisan_2008}. Emerging research has focused on party ID as a fundamental social identity, contributing to the polarized political environment by stimulating negative affective partisanship between Democrats and Republicans and reinforcing in-group/out-group hostilities among race/ethnic groups \citep{iyengar_origins_2019}. They note that, unlike race and religion, social norms do not govern inter-group relations among partisans --- it is socially acceptable to spew contempt for one another and use derogatory labels to do so. Indeed, such comport characterizes the quotidian behavior or the political elite on social media and speeches. Increasingly, this is thought to be a factor in the rising political polarization in the U.S. \citep{west_partisanship_2020, abramowitz_united_2019}. 


\subsection{Trump, Race, and Realignment}

Following Trump's accession to the White House in 2016, the topic of realignment experienced a renaissance in research, particularly racial realignment. This came in response to widespread media pronouncements that Trump had redrawn the electoral map (which would necessarily imply a realigning election), and exit polling that revealed a large defection of White voters to the Republican camp. The majority of researchers rejected the conclusion that 2016 was a critical election. Some researchers argued that this shift was simply a continuation of decades-long realignment of White voters into the Republican camp, and utterly unremarkable \citep{carnes_white_2021, johnston_was_2017, bartels_analysis_2016}. Others contended that the unusually large shift among White voters the rust belt demanded a more robust explanation. Various methodological approaches identified disproportionate vote-switching by voters who had supported Obama in 2012, and tentatively qualified 2016 as a deviating election \citep{morgan_trump_2018, reny_vote_2019, mutz_status_2018}. For example, \citet{morgan_trump_2018} found that 28 percent of Trump support in 2016 came from Obama voters, a conversion effect (assuming 5 to 7 percent of those were swing voters), or nonvoters, a mobilization effect, versus 16 percent of Clinton's support from the same. Assessments of the election were all over the map, defining the election as a continuation of a secular realignment along ideological \citep{abramowitz_united_2019}, cultural \citep{highton_cultural_2020}, or racial fault lines \citep{kitschelt_secular_2019}. The time frames for these various realignment frameworks are similarly disharmonious. 

This exposes a fundamental issue in the debate surrounding realignment theories. Just as the critical election paradigm can disputed with evidence of partisan change predating the electoral cycle or lingering long after its denouement, so too the temporal endpoints of secular realignments can become moving goalposts. Both elections and partisanship are subject to age, cohort, and period effects, complicated by a lack of longitudinal surveys and voter secrecy. Elections are snapshots in time. Partisanship evolves. The stability of partisan identification means that it is sticky. The psychological costs of casting a ballot across the aisle are far lower for a voter than abnegating partisanship. Partisan change generally starts with cross-voting for this reason. Politicians who switch parties forfeit institutional resources, such as high-profile committee chairs for members of congress, and typically fare worse in elections thereafter \citep{grose_electoral_2003}. Early in the Southern realignment, although White Democrats broke with their party to support Republican presidential candidates, they continued voting for Democrats at lower levels in part because the relatively weak Republican party often could not recruit or field candidates \citep{black_rise_2009}. Critics have rejected the entire framework of realignment elections on the grounds that elections cannot be neatly dichotomized and that partisan change is an inherently gradual process \citep{mayhew_electoral_2000, carmines_issue_1989}.   

Elections have consequences, and some are more consequential than others. Relaxing the assumptions, analytical thresholds, and typologies of the critical election literature, it remains clear that the balance of partisan power does change, and that electoral cycles are related to realignment. The elections of 1964 to 1972 unquestionably standout as crucial to the long, secular reconfiguration of Southern partisanship, even if consensus on the timing and velocity has been elusive. The process transformed the landscape of political parties from sectionalism into the contemporary national coalitions revolving around such sociocultural issues as abortion, civil rights, religion, and free speech \citep{adams_abortion:_1997, carmines_role_2002, carsey_changing_2006}. Without rejecting the roles that Southern economic modernization, cultural values, elite polarization, and ideology played in the overall transformation, it is generally agreed not only that race was central to realignment, but that race has since evolved into a defining feature of American partisanship, one that subsumes class divisions and overshadows economic issues, and is thus pivotal in the debate over prospects that the Democratic or Republican parties will achieve dominance in coming years \citep{carmines_issue_1989, carmines_mobilization_1995, miller_activists_2003, valentino_old_2005, knuckey_ideological_2001, phillips_emerging_2015, judis_emerging_2002}. Developing evidence that racial realignment may experienced a critical moment in the 2016 and 2020 electoral cycles would inform the realignment literature and add empiricism to the debate's renaissance.          

One way to recover the utility of the concept of critical realignment elections is to relax the many assumptions that have built up in the literature over the years --- Mayhew lists 11 in his critique. A straightforward approach is to return to the original source: V.O Key, who introduced the concepts of critical and secular partisan realignments \citep{key_jr_theory_1955, key_jr_secular_1959}. Key recognized that elections cannot be neatly dichotomized and that partisan change is an inherently gradual process. He defines a critical election as one in which the outcome is a sharp and durable realignment, but with the caveat that in practice it is unlikely that actual elections will conform to an ideal type, nor will it apply to all of the ``radically varying'' types of behavior among groups of voters \citep{key_jr_theory_1955}. Key expected secular change to precede a critical election, and hypothesized that cumulative change could build toward such an event. He also recognized that realignments do not happen overnight after an election day. This suggests a spectral distribution rather than a dichotomy. At one end are elections in which partisan preferences remain constant. At the other end, large majorities of voters change their partisan allegiance abruptly without warning. In theory, all elections fall somewhere on this spectrum, which, in the interest of parsimony, assumes stable turnout among groups. In practice in the U.S. few voters change their presidential party vote from one election to the next, meaning elections are near to the constant partisanship endpoint. If there is any movement toward the other extreme, it could be the result of swing voters (vote regularly, often switch), nonvoters or surge voters (vote rarely but were activated for a certain election), or conversion. Given that swing voters have steadily declined as a share of the electorate, an increase in their share of ballots would signal that partisan voters had dealigned (but not necessarily enduring enough to qualify as a realignment). If nonvoters or one-off surge voters increased their share of the ballot tally, that would signal a mobilization effect, with the same caveat. Any increase net of that would signal conversion. 

\subsection{Implications and Importance}

The aim of this research is not to propose a model for this spectrum. Rather, it is not assess the role or race/ethnicity in Georgia's recent elections with the objective to develop evidence of partisan realignment in the 2016 and 2020 cycles. In terms of critical election theory, the goal is to apply a more flexible framework in order to accommodate more relevant conclusions. There is a need in the literature to move in this direction. \citet{kitschelt_secular_2019} describe the 2016 election as an ``acceleration'' of existing realignment trends; \citet{burnham_current_2016} characterizes it as ``a very special type'' of critical election; \citet{reny_vote_2019} found evidence of racial realignment, but stopped short of making that conclusion. By estimating voting patterns by race/ethnicity over a reasonable number of presidential elections in Georgia (2000-2020), this study aims to build evidence that racial realignment hit a critical juncture in 2016 and 2020. Recognizing that critical conversion cannot be confirmed until 2024 at the earliest, the endeavor is to estimate voting by race in the State of Georgia, which had a Cook Partisan Voting Index of +7 Republican going into the 2016 elections and handed Democratic candidate Biden a win in 2020. This provides sub-national evidence of race/ethnicity-based vote-switching detected in 2016 and 2020 at the national level \citep{reny_vote_2019, mutz_effects_2022}. Despite significance of Biden's Georgia upset, research of the state's electorate is lacking. In 2016, \citet{hill_not_2021} estimated the contribution of turnout and conversion effects in the 2016 election in several battleground states, including Georgia. They determined that the average 2.7-percent precinct swing toward Hillary Clinton versus 2012 election results was explained almost entirely by turnout differences from 2012, and that net conversion contributed more votes to Trump than to the Democratic ticket. There is an outstanding need to analyze the relative importance of turnout and conversion by race/ethnicity and extend the research to 2020. 

The present chapter considers dealignment and realignment within the conversion thesis. The weakening of partisan zeal that accompanies dealignment could prime voters for conversion \citep{stanley_partisan_1988}. Minor parties can serve as `halfway houses' for voters transitioning between major parties \citep{hammond_minor_1976}. Since it may be premature to judge whether the partisan shifts of 2016 and 2020 in Georgia represent dealignment, realignment, or both, the analysis tests for evidence of either dynamic. Whereas marcopartisanship may shift as a result of demographic change that remakes the electorate, or through the mobilization of voters who previously did not vote, this study hypothesizes that the victory of the Democratic presidential candidate and two senators in Georgia's 2020 election signals a realignment with White voters switching parties in numbers large enough to rebalance the state's partisan profile. 

It does so in the spirit of updating research into the Emerging Democratic Majority hypothesis. Political scientists and Republican strategists alike have recognized the ineluctable arithmetic of demographic change in the American electorate for decades \citep{judis_emerging_2002, teixeira_demographic_2010, frey_karl_2014}. If such a coalition emerges, it will be felt in the U.S. Electoral College at the state level. Journalists have been focused on the issue of battleground states for more than a decade. Weeks before joining the Applied Demography program, as a journalist I published newspaper article on the Democratic party's Battleground Texas, a campaign to elevate the party's competitiveness in the state by leveraging status as a majority-minority population. In an interview, I asked Teixeira what he thought about Texas's battleground prospects. ``Everyone's focused right now on Texas flipping, but Georgia's the state to watch. It will flip first'' \citep{Personal_Communication_Teixeira}. At least about that, he was right.  


\subsection{Hypotheses}

\begin{hypothesis}In the 2016 and 2020 electoral cycles there has been a significant defection of White voters to the Democratic ticket in Georgia.\end{hypothesis}

\begin{hypothesis}
The effect of changing White partisan preference contributed enough votes to the Democratic ticket to change the balance of the elections.  
\end{hypothesis}

\begin{hypothesis}
The magnitude will be such that even if minority vote patterns are found to have remained stable or trended to Republican ticket support, partisan switching by White Voters were the decisive factor in the 2020 victory.  
\end{hypothesis}

% \begin{hypothesis}
% The movement of White voters in Georgia away from the Republican party will show an association with rising national and local political conflict. 
% \end{hypothesis}


\section{Data and Methodology}

Despite the explosive growth of data that is collected on voters by surveys and other sources, at sub-national geographies it is still problematic to estimate a time series going back more than a few four-year presidential cycles. One potential avenue is to utilize a national survey employing a disaggregation technique. Such a technique is implemented in this chapter to estimate the party preference of voters for presidential candidates at different demographic and geographic levels. Using multilevel regression with poststratification (MRP), partisan voting choice is modeled as a function of select geographic and demographic variables in a national electoral survey. The MRP methodology enables the probability of voting preferences for different demographic groups in Georgia to be estimated using data from a national survey using sample sizes that are too small for conventional logistic regression techniques to . As a validation strategy, estimates are produced for observed state-level results first, then for race/ethnic groups that are compared with exit polls. 

\subsection{Multilevel Regression with Poststratification}

MRP combines small-area estimation methodologies with poststratification by modeling the response variable conditional on demographic variables \citep{gelman_poststratification_1997}. Proposed originally as an application to disaggregate national opinion polls at geographic levels where sample sizes are inadequate for conventional statistical approaches, MRP has been demonstrated to be effective for a wide range of political science research initiatives, including electoral forecasting, public opinion research, and estimating voting preferences and behaviors \citep{gelman_poststratification_1997, ghitza_deep_2013, gelman_mister_2013, lax_gay_2009, park_bayesian_2004}. In recent years, the model-based approach has been adopted by such fields as demography, psychology, and public health \citep{claassen_improving_2020, wang_using_2018, wang_comparison_2017, zhang_multilevel_2014, kennedy_know_2020}. 

 Bayesian hierarchical models are fit on  a national elections survey time series for each electoral contest of interest using sociodemographic variables present in Census data. The variables are constructed as categorical variables with varying intercepts so that the model captures their conditional effects. As such when race and education are combined in the prediction step, their interdependence is accounted for, rather than treating them as independent effects. According to Bayesian principles, models are specified with a prior expectation of outcome probabilities, and then these probabilities are updated by sampling repeatedly from observed data. The predictions settle into likelihood distributions also known as posterior distributions. These prediction matrices can be simulated on Census data featuring variables that match those in the model. In this case the logistic odds for voting for a Democratic candidate for president are estimated for combinations of the independent variables.

 To execute the analysis, a hierarchical model is built with a set of categorical demographic and geographic variables that are present in census data -- sex, age, race/ethnicity, educational attainment, marital status, and state of residence. These indexing variables are set at the individual level, and theoretically relevant contextual variables at higher levels. The demographic and geographic intercepts are allowed to vary randomly so as to pool information among all census strata, strengthening estimates for each category and all possible combinations. The mean posterior distributions of model parameters are weighted by the population size of the cell of interest, even for cell types not present in the survey (or other source of) data, a common occurrence and frequent obstacle to classic poststratification \citep{gelman_data_2006}. Estimates for missing demographic/geographic categories utilize information from the effects estimated for each group and augment it by drawing on the overall grand mean of the hierarchical model. In the present study, the Georgia sample sizes for each election in the ANES cumulative data file are insufficient for statistically valid direct survey estimates at the state level, and even less so for demographic subgroups, but robust MRP estimates can be generated for Georgian voters by race and educational attainment so long as the population sizes of the corresponding cells are known.   

The outcome variable is a binary response coded `1' if the respondent states he or she voted for the Democratic presidential ticket in presidential cycles, or the top-of-ticket candidate in others, and `0' if their ballot was cast for the Republican or minor party candidate.  Age, sex, race, education, marital status, and state of residence are coded into categorized variables as described below. The results are then poststratified on Census counts of the citizen-voting age population and reweighted for each demographic/geographic combination of interest. A time series is created that reconstructs the likely voting preferences of Georgian voter groups.  

The modeling approach employs multilevel models that are integrated via Bayes theorem, a framework that explicitly incorporates both uncertainty and expert knowledge into the probability of a particular outcome by specifying prior probability distributions for model parameters. For this study, models are constructed with the an R-based interface to the Stan Bayesian programming language \citep{goodrich_rstanarm_2020}. The typical Bayesian workflow involves model fitting, iteration, diagnostics, evaluation, and inference. The first stage involves assessing the problem at hand to formulate some quantitative propositions about potential outcomes based on past data or knowledge, and framing a responsive model. Then, prior assumptions regarding the potential shape of the distribution of the inputs, dependent associations, and potential outcomes are built into the parameterization. This specification of probabilities in the Bayesian approach essentially nudges model fitting in the direction of an educated guess, along with an explicit likelihood evaluation to define the threshold that must be reached before  empirical evidence will support prior expectations. In the modeling step, this probability is updated by iterative simulations to generate a prior distribution, which is mathematically assessed against the probability distribution specified during model design, ultimately producing a conditional probability with an explicit range of uncertainty. Bayesian inference is computationally expensive and misspecified priors or parameters typical result in a failure of the model to converge --- though in theory data updating would eventually generate robust estimates if a the sampling algorithm were allowed to continue.  

As a rough example, with a presidential election approaching, a research question could be formed regarding the preferences of male and female voters for the Democratic candidate for president. A binary outcome of support or opposition could then be defined (excluding third-party tickets), generating a Bernoulli probability distribution expressed in the form ${y}_{j}\sim Bernoulli\left(n_j,{\theta}_{j}\right)$. ${\theta}$ is the overall average estimated voting preference, and $y_j$ is the outcome in each cell $j_{1}...j_{102}$ from $J$ = (2 sexes and 51 states plus DC). In this case, $n_j$ is the number of votes by sex and state, and ${\theta}_{j}$ the probability. Based on prior voting patterns from previous elections, a model could be expressed that expected women to be more likely to support a Democrat than men by some degree, perhaps with a varying likelihood for each by state. These priors beliefs suppose differing probabilities by sex, but might also recognize the uncertainty implicit in measuring true levels of support. Strong or weak prior beliefs inform to a greater or lesser degree the assumptions placed on the covariates, the model structure, the uncertainty surrounding the data, and numerous other tuned parameters. In the next step, observed data is introduced by fitting the model on a weekly poll that asks each respondent for their gender identification and party vote intention. If data accrues that females tend to favor the Democratic candidate, that evidence is weighed against the likelihood that this constitutes actual support, given the chance that the pollster is not generating an accurate representation. However, if weekly polls report consistently higher level of support for a Democratic ticket than male respondents, even considering the odds of survey bias, prior beliefs are updated, and eventually the gap between sexes by state can be estimated with a stated degree of certainty.

\subsection{Data}\label{sec:anesracechapter}

The primary survey data come from the ANES Cumulative Data File (CDF) (see Table \ref{tab:anestable1}), the longest-running systematic study of U.S. elections \citep{brader_american_2021}. Principal investigators the the University of Michigan and Stanford University have harmonized variables and weights across survey years for the pooled time series. Some sample years include panel respondents who are interviewed in consecutive elections, however the CDF is not a longitudinal design, but rather an ongoing series of cross-sectional surveys. Angus Campbell and Robert Kahn pioneered the first national electoral study in the nation's history at the renowned Survey Research Center (SRC) at the University of Michigan, and has been conducted at least every four, and in some periods every two years, ever since. The National Science Foundation has funded the project since 1978.   

\begin{table}[ht]
\centering
\begin{threeparttable}
\caption{Georgia ANES survey respondents who reported voting}
\label{table:ANES-respondents-race}
\begin{tabular}{rrrrr}
  \hline \\ \vspace{0.45em}
 & White & Black & Hispanic & Other \vspace{0.45em}\\
\hline
  2000 &  12 &   5 &   1 &   2 \\ 
  2004 &   2 &   5 &   0 &   0 \\ 
  2008 &  29 &  23 &   0 &   1 \\ 
  2012 &  61 &  43 &   7 &   8 \\ 
  2016 &  54 &  15 &   4 &   6 \\ 
  2020 & 115 &  39 &  11 &  17 \\ 
   \hline
\end{tabular}
{\footnotesize Source: ANES Time Series Cumulative Data File}
\end{threeparttable}
\end{table}

A poststratification table is also used in the analysis, built from population detail for each category of state, age, sex, race, educational attainment, and marital status for presidential elections from 2000 to 2020. Consistent with ANES, the universe is the citizen voting-age population (CVAP) of the United States. Census data to construct the table come from the IPUMS-CPS basic monthly samples and the biennual November voting supplement \citep{ruggles_ipums_2021}. Variables are harmonized across the samples by IPUMS, and weights from the respective surveys are used to estimate the population of the cells. The weights are representative of the citizen voting-age population. Each census survey was aggregated into the cells matching the model specification. The most common demographic variables that partisan preference models seek to control for are present in the census datasets. The resulting poststratification table contains White, Black, Hispanic, and Other citizen-voter age population counts for all states. 

\newpage
\begin{table}
\small
\centering
\begin{threeparttable}
\caption{ANES CDF 2000-2020, summary statistics} 
\label{tab:anestable1}
\begin{tabular}{llllllll}
  \hline \\ \vspace{0.45em}
  & 2000 & 2004 & 2008 & 2012 & 2016 & 2020 \\ 
  \hline \\ 
   n (\%) & 1,807 & 1,212 & 2,322 & 5,914 & 4,270 & 8,280 \\ 
       & &  & & & & &\\ 

    AGE& &  & & & & &\\ 
    18-24 & 146 (8.1) & 127 (10.5) & 221 (9.7) & 511 (8.7) & 328 (7.9) & 408 (5.1) \\ 
    25-34 & 306 (17.0) & 205 (16.9) & 404 (17.7) & 881 (15.0) & 710 (17.1) & 1,228 (15.5) \\ 
    35-44 & 432 (24.0) & 215 (17.7) & 413 (18.1) & 887 (15.2) & 655 (15.8) & 1,378 (17.4) \\ 
    45-54 & 336 (18.7) & 238 (19.6) & 471 (20.7) & 1,107 (18.9) & 688 (16.6) & 1,202 (15.2) \\ 
    55-64 & 263 (14.6) & 219 (18.1) & 360 (15.8) & 1,256 (21.5) & 817 (19.7) & 1,474 (18.6) \\ 
    65-74 & 173 (9.6) & 120 (9.9) & 228 (10.0) & 851 (14.5) & 623 (15.0) & 1,449 (18.3) \\ 
    75+ & 142 (7.9) & 88 (7.3) & 180 (7.9) & 361 (6.2) & 328 (7.9) & 793 (10.0)\vspace{0.5em} \\ 
    SEX& &  & & & & &\\
    Male & 790 (43.7) & 566 (46.7) & 999 (43.0) & 2,845 (48.1) & 1,987 (47.1) & 3,763 (45.8) \\ 
    Female & 1,017 (56.3) & 646 (53.3) & 1,323 (57.0) & 3,069 (51.9) & 2,231 (52.9) & 4,450 (54.2) \vspace{0.55em}\\ 
  RACE/ETH    & &  & & & & &\\ 
 White & 1350 (75.5) & 848 (71.2) & 1,138 (49.5) & 3,518 (59.7) & 3,038 (71.7) & 5,963 (72.9) \\ 
    Black & 198 (11.1) & 176 (14.8) & 560 (24.4) & 1,027 (17.4) & 397 (9.4) & 726 (8.9) \\ 
    Hispanic & 140 (7.8) & 112 (9.4) & 523 (22.8) & 1,009 (17.1) & 450 (10.6) & 762 (9.3) \\ 
    Other & 101 (5.6) & 55 (4.6) & 77 (3.4) & 336 (5.7) & 352 (8.3) & 727 (8.9) \vspace{0.5em}\\ 
EDUC     & &  & & & & &\\ 
  $<$ H.S. & 64 (3.6) & 37 (3.1) & 100 (4.3) & 127 (2.2) & 41 (1.0) & 376 (4.6) \\ 
    High School & 635 (35.3) & 429 (35.4) & 1,003 (43.5) & 1,937 (33.1) & 1,056 (25.0) & 1,336 (16.4) \\ 
    Some Col. & 545 (30.3) & 384 (31.7) & 712 (30.8) & 1,958 (33.5) & 1,499 (35.4) & 2,790 (34.2) \\ 
    BA/BS & 556 (30.9) & 362 (29.9) & 493 (21.4) & 1,824 (31.2) & 1,635 (38.6) & 3,647 (44.8) \vspace{0.5em}\\ 
 MARITAL     & &  & & & & &\\ 
  Married & 935 (52.1) & 625 (51.6) & 975 (42.2) & 2,941 (49.8) & 2,142 (50.4) & 4,322 (52.6) \\ 
    Single & 348 (19.4) & 276 (22.8) & 404 (17.5) & 1,184 (20.1) & 786 (18.5) & 1,470 (17.9) \\ 
    Other & 510 (28.4) & 310 (25.6) & 929 (40.3) & 1,779 (30.1) & 1,318 (31.0) & 2,432 (29.6) \vspace{0.5em}\\ 
   \hline
\end{tabular}
\end{threeparttable}
\end{table}
\newpage

\subsubsection{Weights}

ANES documentation recommends the use of CDF weights for analyses including regression, as some of the weighting factors, such as number of telephones in respondent household, are not available as variables \citep{debell_how_2010}. However, Bayesian models do not incorporate survey design weights and Stan developers recommend against their use \citep{stan_development_team_stan_2021, hanretty_introduction_2019, gelman_struggles_2007}. The poststratification of hierarchical regression estimates automatically adjusts sample responses to the target population. Researchers have demonstrated the utility of MRP to outperform survey weights and to correct highly nonrepresentative samples \citep{downes_multilevel_2018, downes_multilevel_2020, wang_forecasting_2014}. Weighting was incorporated into the model by specifying the frequency of collapsed rows obtained from the aggregation of voting preferences for the proportional binomial models.  

The poststratification table employs survey weights to build cell-level counts for all possible cross-tabulations of the geographic and demographic variables employed in modeling. Person-level weights (PERWT) from IPUMS-USA are used to calculate how many persons in the universe are represented in each cell. These weights are the same as the CPS voter sample weight (VOSUPPWT), representative of the citizen voting-age population (CVAP). As the analysis is run on subsamples, first a survey design object is built using the Survey package in R, and then the object is subset by year, age, and citizenship  \citep{lumley_survey_2020}. The population of the cells are calculated a from the resulting survey-weighted table.    

In addition to the survey and census datasets, state-level contextual variables were developed from various sources. These sources are referenced in the variable descriptions and provided with additional detail in the appendix.  

\subsubsection{Survey Variables}

\begin{itemize}

\item \textbf{Race/Ethnicity}. Race is the primary categorical variable of interest among the MRP indexing variables that create the poststratification cells. In all survey years, race and ethnicity is self-reported by respondents. Hispanic is included as a race category. The re-coded race variables in the cumulative time series data file define White as `non-Hispanic white', Black as `non-Hispanic black', Hispanic, and `other' across all years.  

\item \textbf{Educational Attainment}. Categories exist for both the surveys and poststratification tables for less than high school, high school diploma, some college, and four-year degree. A fifth level is available in the data, indicating an advanced degree, but testing indicated that the additional category did not impact the overall trend in voting by race and educational attainment, and thus was not included in the final results. 

\item \textbf{Sex}. A dummy variable for sex specifies females as 0.5 and males as -0.5. 

\item \textbf{Age}. Age, an important characteristic when measuring voting patterns, is a categorical variable used for poststratification and divided into seven ten-year groups, following the groups in the ANES codebook: 18-24, 25-34, 35-44, 45-54, 55-64, 65-74, 75+. A small number of 17-year-old respondents are recoded as 18, as respondents would have been that age at election time to be eligible for the ANES, and because the poststratification table would be incorrectly weighted if all individuals aged 17 were grouped into the cell. 

\item \textbf{Marital Status}. Marriage is known to be an important predictor of voting behavior, increasing the likelihood of turnout; Republicans are also more likely than Democrats to be married, particularly among women \citep{wolfinger_family_2008, gershkoff_marriage_2017}. Marital status is defined as never married, married, or other.

\item \textbf{State}. States are modeled as geographic categorical variable as part of the poststratification strategy to estimate results at various demographic and geographic combinations. 

\end{itemize}



% \item \textbf{Income}. Income, also a demographic indexing variable, has been strongly associated with partisanship in previous research. ANES harmonizes household income by creating five categories: 0 to 16 percentile, 17 to 33 percentile, 34 to 67 percentile, 68 to 95 percentile, and 96 to 100 percentile.

% \item \textbf{Party ID}. Partisanship is represented by a three-level factor variable coded as Democratic, Republican, or Independent. ANES recodes the original 7-point scale to three and independents who report leaning toward one party are collapsed into that category. 

% \item \textbf{Racial Resentment}. The racial resentment index is created by adding the scores of four items in the ANES conventionally used for this purpose by researchers and dividing by 20. The items ask respondents if Blacks should improve their socioeconomic conditions like Italian and Irish immigrants without any special favors from the government, if the legacy of slavery is an impediment to Black upward mobility; if Blacks have gotten less than they deserve, and if Blacks need to try harder. For modeling purposes, the index is z-scored. The sub-index variables are only available from 1986 to 2016. \citep{smith_dynamics_2020}

% \item \textbf{Authoritarianism}. The authoritarian index is created by adding the scores of four items from the child rearing scale: What is more important in a child, independence or respect for elders; curiosity or good manners; obedience or self-reliance; and to be considerate or well behaved. The scale is correlated with the F-scale and the Right-Wing Authoritarian scale, and has been used to evaluate respondents' authoritarian proclivities. For modeling purposes, the index is z-scored. The sub-index variables are only available from 1990 to 2016. \citep{smith_dynamics_2020}


\subsubsection{Higher-Level Variables}

\begin{itemize}

%  \item \textbf{Polarization}. Societal conflict is proxied by the DW-NOMINATE measure of political polarization in the House and Senate, calculated for each congressional roll-call vote by \citet{lewis_voteview_2021, lewis_rvoteview_2015}, and then averaged for the Democratic and Republican parties by state. The measure represents the distance between party ideological mean scores on the NOMINATE1 dimension. This measure tests the hypothesis that racial partisan realignment has increased conflict, creating a superordinate threat that has led higher-educated White voters to seek better intergroup relations through support for the multiracial Democratic party coalition.         
% \begin{figure}
% \centering
% \includegraphics[width=.75\textwidth]{pics/ga_us_pol.png}
% \caption{Polarization score in Congress}
% \label{fig:pol}
% \end{figure}
\item \textbf{PVI}. The partisan voting index is calculated based on the average Democratic vote share of each state in the previous two presidential elections versus the national average. A PVI score of .02 PVI in 2016 signifies that the partisan profile of the state electorate for that election is two basis points more Democratic than the national average, based on results from the 2008 and 2012 presidential elections. The addition of this covariate smooths the differences for the overall partisanship profile of individual states and oscillations in national trends, and is commonly used in MRP analyses involving elections. 

\textbf{Minority Population}. The importance of race/ethnicity on partisan preference is introduced into model fitting by calculating state-level population composition from the U.S. Census Bureau's annual estimates program. A scaled percent of Black and Hispanic share of state population is applied. 

% The variable is intended to test for a relationship between the proportion of minority populations in the U.S., a theoretical underpinning of the racial threat hypothesis. Racial threat is generally considered to be activated by proximity of sizeable non-White groups, thus the county-level implementation is preferred. For models that are fit on the ANES survey, however, the lowest geographic identifier is that state. This analysis also includes percent minority squared in order to test for a curvilinear relationship. 

% \textbf{Minority Growth}. Scholars of U.S. racial group conflict have also used rate of increase of a minority in a given area \citep{blalock_economic_1956}. Since population estimates are produced annually, the year-on-year rate of growth can be calculated for all categorical minority population variables. These is intended to test for a relationship between partisan outcomes and rate of minority population growth, another potential threat activator.  

% \textbf{Gross Domestic Product}. U.S. economic growth at the national level, a strong predictor of presidential approval, controls for a candidate effect \citep{abramowitz_forecasting_2008}.

% \textbf{Unemployment}. A national unemployment rate is used to account for the potential economic threat associated with competition for jobs among race/ethnic groups.

\end{itemize}


\subsection{Models}

The strategy calls for fitting hierarchical generalized linear mixed-effect models on the ANES survey for each electoral year, using the same specification for each cycle, when possible. For each election, two models are fit. The first estimates the probability of a voter casting a ballot for a major party, and the second estimates the probability that, given a major party vote, the ballot is cast for the Democratic candidate. A categorical race variable describes four groups: White, Black, Hispanic and Other. State, race, age, marital status and education are given varying intercepts with double and triple interactions. Hierarchical models fit on multiple groups with random effects confront issues with computational stability and convergence of the sampling chains can be a challenging hurdle for the technique. Testing indicated that balance is important. Simple model structures are much more likely to converge, but mathematically complex double- and triple-interactions contribute more information for partial pooling and are generally recommended \citep{ghitza_deep_2013}. Weakly informative priors are modeled as normal(0,1), tighter than the rstanarm default of normal(0, 2.5). Likewise, the covariance matrix is assigned a prior of exponential(scale=0.5), versus the default of exponential(scale=1). 

It is common practice to include selected contextual covariates along with the group indexing variables to smooth out group differences. In all models, a partisan voting index is z-scored and implemented as a continuous state-level indicator. This helps account for state differences, candidate appeal net of partisanship, and swings in national mood. Black and Hispanic population percentage by state are z-scored and modeled when possible. The major-party outcome is specified by defining a dummy variable $voted\_major$ and restricting respondents to registered voters how responded to the question of whether or not they voted. If a respondent in this universe confirmed voting for either the Republican or Democratic candidate, they are coded as `1', else `0'. Two-party estimation is restricted to voters who voted major, and the variable $given\_dem$ is defined as `1' for voting for the Democratic candidate at the top of the ticket, else `0'. The number of Democratic votes out of the total number of voters in each cell are summed and operationalized as a trial success, while voting for Republican or third party is the calculated by subtracting from all votes in a cell, producing $dem$ versus $rep$. To reduce the considerable computational burden involved with sampling from the joint distribution of multiple group-level varying intercepts with double and triple interactions, the outcome is aggregated for each electoral year by state ($st$), age ($a$), sex ($s$), race ($r$), education ($e$), and marital status ($m$) and transformed into a proportional binomial model. 

There are $K$ numbers of $J$ censal indexing factors $j_{1}=\{1,\ldots ,J_{1}\}$ through $\,j_{k}=\{1,\ldots ,J_{K}\}$, with varying intercepts $\alpha_{1}^{k}\ldots \alpha_{J_{k}}^{k}$. For example, if the $j$th respondent is Hispanic, $\alpha_{\rm r[j]}^{\rm race}$ takes the value {$\alpha_{\rm r[Hisp]}^{\rm race}$}. The general hierarchical model takes the form:
\begin{gather*}
\text{y} \sim \text{Binomial} (n, p)\\
\text{voted\_major}  \sim maj_i = 1 (n_i, p_m)\\
\text{voted\_dem}  \sim dem_i = 1 (n_i, p_d)\\
\end{gather*}
where: \newline 
$y_m$ is a vote for a major-party candidate;\newline
$y_d$ is a vote for the Democratic candidate; \newline
$n$ is the population in a demographic cell; \newline 
$p_m$ is the probability of voting for either a Democrat or Republican \newline 
$p_d$ is the probability of voting for the Democratic candidate for president, \newline given the probability of $p_m$. The equation is expressed as:
\begin{gather*}
\text{logit} (p_i) = \alpha_{\rm a[j]}^{\rm age}
+ \alpha_{\rm s[j]}^{\rm sex}
+ \alpha_{\rm r[j]}^{\rm race}
+ \alpha_{\rm e[j]}^{\rm educ}
+ \alpha_{\rm m[j]}^{\rm marital}
+ \alpha_{\rm st[j]}^{\rm state}
+ \alpha_{\rm re[j]}^{\rm region}
\end{gather*}
\text{and the varying coefficients are assigned independent prior distributions:}
\begin{align*}
\alpha_{\rm a[j]}^{\rm a} & \sim N(0,\sigma^{\rm age})\\
\alpha_{\rm s[j]}^{\rm s} & \sim N(0,\sigma^{\rm sex})\\
\alpha_{\rm r[j]}^{\rm r} & \sim N(0,\sigma^{\rm race})\\
\alpha_{\rm e[j]}^{\rm e} & \sim N(0,\sigma^{\rm educ})\\
\alpha_{\rm m[j]}^{\rm m} & \sim N(0,\sigma^{\rm marital})\\
\alpha_{\rm st[j]}^{\rm state} & \sim N(0,\sigma^{\rm state})\\
\alpha_{\rm re[j]}^{\rm region} & \sim N(0,\sigma^{\rm region})
\end{align*}
% Pr\left(Y_i = \mbox{Dem} | Y_i\in\{\mbox{Dem, Rep}\}\right) = \mbox{logit}^{-1}(\alpha_0 + \alpha_1(\mbox{pvi}) + \alpha_{j[i]}^{\mbox{state}} + \alpha_{j[i]}^{\mbox{edu}} + \alpha_{j[i]}^{\mbox{sex}} + ...)
 
\subsubsection{Diagnostics}

Models are fitted with the Stan-based R package `rstanarm' \citep{goodrich_rstanarm_2020}. The stan\_glmer function generates \textit{stanreg} objects in R \citep{r_core_team_r_2021}. The Rstanarm interface includes a range of diagnostic functions. Model quality comparison during testing and development was conducted using the CRAN R `Performance' package, which calculates such as metrics as conditional and marginal R-squared and root mean squared error (RMSE) \citep{ludecke_performance_2021}. The output was also converted into a shinystan object for the purpose of visual and numeric diagnostics \citep{gabry_shinystan_2018}.

\begin{figure}[ht]
\includegraphics[width=.45\textwidth]{UTSAthesisPackage/pics/anes5_2008_rhat.png}
\includegraphics[width=.45\textwidth]{UTSAthesisPackage/pics/msce_error.png}
\caption{Diagnostics example: $\hat{R}$ \& MCSE v. posterior SD}
\label{fig:anesrhat2008}
\end{figure}

\noindent The shinystan interface warns of potential errors with models, flagging the presence of such issues as divergent sampling chains, sufficient mixing of between- and within-chain estimates, or small effective sample sizes. The presence of post-warm up divergences, which may signify the results are biased, can be particularly troublesome in an MRP specification due to the typical usage of numerous group-level random intercepts. This was overcome by forcing the Stan sampling algorithm to take use very small step-sizes by setting adapt\_delta to between 0.99 and 0.999, substantially smaller -- and slower -- compared to the Stan default of 0.80 \citep{stan_development_team_stan_2021}. Once sampling chains converge, iterative adjustments were tested using model performance metrics. Additional diagnostics within the shinystan interface also suggested specific modifications and provide further validation of the goodness of fit. A prominent diagnostic approach relies on the $\hat{R}$ statistic, which measures variance of pooled samples and should fall below 1.10, is displayed numerically and graphically (figure \ref{fig:anesrhat2008}). Statistics and plots for Monte Carlo standard error, effective sample sizes ($n_{eff}$, $n_{eff} / N$), and autocorrelation can also lend additional evidence to successful sampling and model goodness. Finally, shinystan generate predictive posterior checks, replicating the original data from the posterior samples. If the model is a good fit, the predicted data will resemble the original data (figure \ref{fig:anes2020ppe}). 

\begin{figure}[ht]
\includegraphics[width=1\textwidth]{UTSAthesisPackage/pics/anes5_2020_distvobser.png}
\caption{Diagnostics example: 2020 ANES Shinystan Posterior Predictive Check}
\label{fig:anes2020ppe}
\end{figure}


\subsection{Poststratification}

This model-based inference occurs at the second stage of MRP. The poststratification table, constructed from the Current Population Surveys and constrained to the citizen-voting age population, is a highly accurate representation of the electorate, and as such reflects the ANES sampling frame. Neither accounts for voter ineligibility due to conditions such as incarceration, felony record, or mental incapacity, the definitions of which vary according to state jurisdiction. 

\vspace{.5 cm}
\textbf{Poststratification variables}
\begin{itemize}
    \item  State: 50 US states and DC ($St = 51$)
    \item  Age: 18-24, 25-34, 35-44, 45-54, 55-64, 65-74, 75+ ($A = 7$)  
    \item  Sex: Female, male ($S = 2$)
    \item  Race/ethnicity: White, Black, Hispanic, Other ($R = 4$)
    \item  Education: Less than high school, high school, some college, 4-year college or more ($E = 4$)
    \item Marital status: Married, single, other ($M = 3$)
    \item Region: Four Census regions plus DC as a separate region ($CR$ = $5$)
\end{itemize}
\vspace{.5cm}


\noindent For poststratification, variables are grouped according to which cell estimates are to be generated, and the population of each cell is calculated using the Survey package to incorporate Census design weights \citep{lumley_survey_2020}. If higher-level covariates are specified in the model, these must be present in the data, generally joined from a dataframe. Cross-tabulation of the categorical variables generates the population cells. Given that census region is modeled but does not occupy separate cells, the maximum number of cells for all age, sex, race, education and marital status combinations, as well as state of residence is:

\begin{center}
$ 51_{St} \times 2_{S} \times 7_{A} \times 4_{R} \times 4_{E} \times 3_{M} =  34,272$ cells\\
\end{center}

The rstanarm package has a pre-rolled \textit{posterior\_epred} function to extract posterior draws of the linear predictor, transformed by the inverse-link function. Draws are sampled from each election year model's posterior distribution to produce expected values for every level of the new poststratification data frame. The mean MRP estimate is weighted by the frequency of the subgroup of interest (e.g. White voters with a college degree in Georgia) in the census data. The probability of voting for the Democratic ticket is predicted as the expected value of the conditional means of each strata, and then weighted by the demographic $d$ and geographic $g$ characteristics of relevant cells. The result is the estimated probability of voting for the Democratic candidate top-of-ticket, given the probability of voting for a major party. 

\begin{align*}
\label{mrp_ps}
\theta^{MRP} = \frac{\sum N_{\rm dg} \theta_{\rm dg}}{\sum N_{\rm dg}}
\end{align*}


\section{Findings}

The results are evaluated in several stages. First, there is a brief explanation of the style of reporting of MRP results as poststratified predictions, often through the use of graphics or maps, instead of conventional results tables. Second, the results are cross-validated by comparing estimated election outcomes to official reports. Third, the estimates of Democratic voting probabilities by race/ethnic group are reported. With these results in hand, the analysis simulates what the macropartisan balance by race/ethnicity was in each election. Using the ballot share of each race/ethnic group according to official Georgia records, the mean MRP estimates are converted into partisan ballot counts by White and non-White voters. At this stage the change in support for the Democratic and Republican tickets are evaluated to show how aggregate support by voting groups evolved, and the net effect for each party by year. This estimates race/ethnic voting change across the period of estimate, which is the primary aim of the analysis. Next, the electoral cycles are simulated using the lowest MRP estimate of the 90\% credible interval, adopting the most conservative approach to test if findings change directions. The lower bounds for White voters is used in the simulations, and the total vote share is controlled to official reported tallies. The balance is redistributed to non-White voters, preserving the observed results totals. Next, to control for potentially disparate turnout effects of individual elections, the turnout rate of registered voters by race/ethnicity is held constant at the mean rate over 2000-2020, and the election outcome is simulated using MRP estimates. To control for voter registration effects, this simulation is repeated using CVAP as the denominator for turnout rates. Under each of these scenarios, the findings are the same: Biden beat Trump in Georgia in 2020 because White voters defected from the Republican party in unusually large numbers. In a final step, partisanship and campaign contact variables are incorporated into the poststratification table in an extension of MRP to generate estimates of mobilization and conversion.                 

\subsection{Model Results}

Although the multilevel regression models produce statistical estimates that can be interpreted independently of poststratification, the MRP workflow does not typically involve the conventional scrutiny of tables presenting coefficients, standard errors, and significance stars. Often, it may not be feasible to directly test hypotheses with a first-stage MRP regression. The poststratification stage is the endgame, and the choice of variables are constrained by availability in Census data, such as the basic sociodemographic variables of age, sex, race, education and marital status. The use of varying intercepts for categorical variables, particularly when multiply interacted, leads to parameter estimates that can easily number in the hundreds, rendering table presentation and interpretation implausible. In contrast, as \citet{ghitza_deep_2013} note, examining fitted estimates on demographic subgroups in the poststratification step, particularly through visualizations, leads to intuitive interpretations, even for those without statistical expertise. This approach is also responsive to calls from leading political science methodologists to eschew the use of dense tables and jargon in favor of simulated outcomes and practical visualizations \citep{king_making_2000, gelman_lets_2002, kastellec_using_2007}. 

In addition, standard errors of Bayesian models, computed as the median absolute deviation (MAD) of the posterior from the median value of draws, do not offer the same interpretation as standard model table applications. The uncertainty of MRP estimates is typically reported as the standard deviation of the posterior draws from the conditional mean of the cell in question, or as a credible interval defined by the posterior distribution between a defined range, with the most common interval for table display of 10 percent at the lower bounds and 90 percent at the upper limit. Standard errors are relevant when conducting model evaluation and selection, along with such indicators as the Monte Carlo standard error, root mean squared error, Bayes R2, and other metrics.  

\begin{table}[ht]
\centering
\begin{threeparttable}
\caption{ANES 2016, selected parameters} 
\label{tab:selpar} 
\centering
\begin{tabular}{lrrrrrrr}
  \hline \\ \vspace{0.5em}
Parameter & Rhat & n\_eff & mean & sd & se\_mean & 2.5\% & 97.5\% \\ 
  \midrule
Black \% (z) & 1.00 & 46,917 & -0.32 & 0.08 & 0.00 & -0.47 & -0.16 \\ 
  Hispanic \% (z) & 1.00 & 41,631 & -0.08 & 0.04 & 0.00 & -0.16 & 0.01 \\ 
  Female & 1.00 & 61,873 & 0.32 & 0.09 & 0.00 & 0.15 & 0.49 \\ 
  Georgia & 1.00 & 31,836 & -0.03 & 0.09 & 0.00 & -0.27 & 0.12 \\ 
  White * GA & 1.00 & 30,284 & -0.03 & 0.10 & 0.00 & -0.30 & 0.15 \\ 
  White * GA * College Degree & 1.00 & 39,626 & -0.01 & 0.09 & 0.00 & -0.21 & 0.16 \\ 
   \bottomrule
\end{tabular}
\end{threeparttable}
\end{table}


Although full model results are not presented in table format for the reasons just described, a small selection of an effects summary is presented in Table \ref{tab:selpar}. The $\hat{R}$ statistic and other indicators reflect robust model results. Percent Black in the population, percent Hispanic, and a male/female dummy operate as fixed effects, while Georgia, both independently and interacted with White voters and White college educated voters, are random effects. Nationally, Black and female voters are more likely to vote Democrat. The mean estimate for Blacks is $-0.32 \pm .08$. As such, if all other variables including the random effects of the group variables were held constant, there is a 90-percent probability that with every 10 percent increase in a state's Black population, a voter is 1\%  $\pm$ 1\%  less likely to have voted for a Democratic candidate for president in 2016. Nationally, female voters were 1\% more likely to vote Democratic, again adjusting for all other covariates. 

These results do not address the research question, nor are they very enlightening. States with large Black populations in 2016 were generally located in the South, which voted solidly Republican that year; women also voted Democrat more than men in the election, according to weighted estimates from the ANES survey. For all other parameters, the credible intervals cross zero, meaning they could have a negative or positive effect, and would not support alternative hypotheses, were there any being tested at this stage. The lack of significant effect sizes is not an issue at this juncture as the primary goal of fitting the models is to produce estimates from the ANES national survey that may be stratified to race/ethnic groups by educational attainment in Georgia. 

\subsection{Poststratified Estimates}

It is useful to begin by benchmarking the MPR estimates against observed results achieved by the Democratic presidential candidates in Georgia. This study focus on election spanning from 2000 to 2020. In the 1990s, the state's partisan profile was still in flux, even as Newt Gingrich led the Republican Revolution of 1994 \citet{black_rise_2009}. White voters were nearly as likely to identify as Democrats, self-described `Conservative Democrats' retained many statewide offices, and the party's candidates were still winning election to the governorship, the U.S. Senate, and the White House,  After 2000, realignment had fully consolidated in Georgia as the Republican party won trifecta control in the 2004 elections, a feat preceded only by South Carolina and Texas among Southern state peers. The period of interest is characterized by the spreading dominance of the Republican party throughout all levels of government, supported overwhelmingly by White adherents, and pursuing increasingly conservative racial policies. As other observers have noted, racial realignment had come full circle, virtually restoring racialized one-party rule to Georgia politics \citep{lupton_dixies_2020, glaser_back_1994}. 

\begin{figure}[ht]
\centering
\includegraphics[width=1\textwidth]{UTSAthesisPackage/pics/gawithCross.png}
\caption{Validating State-Level Results, 2000-2020}
\label{fig:ga-cross}
\end{figure}


The MRP estimates approximate the observed totals closely, with an average mean error of 1.6 percent. Table \ref{tab:MRP-GA} presents MRP results alongside the percentage of Georgian ballots received by the Democratic candidate for president. The ANES survey estimates are known to suffer from over-reporting bias. At the national level, weighted estimates of two-party share produce Democratic margins two or three points higher than recorded popular vote. The Democratic party began to rebound from the GOP takeover immediately following its trifecta. Although Biden's victory was hailed as an upset, it more accurately represented a watershed moment in a trend years in the making. In fact, Clinton's 2016 run achieved a higher two-party share than Obama in 2012, and was on par with his strong 2008 performance in Georgia. Biden's victory, while certainly an appreciable improvement over recent Democratic tallies, did not surmount the 50-percent threshold (though it did by a razor-thin margin on a two-party basis), and was in line with recent Democratic outcomes in Georgia. However, while the observed results illustrate the increase in electoral competitiveness, they say nothing about the demographics underlying the party's turnaround. The MRP estimates, having been cross-validated with the observable state results, relate that story by poststratifying the posterior expectations at the race/ethnic strata.  

\begin{table}[ht]
\centering
\begin{threeparttable}
\caption{Georgia Presidential Elections, MRP Democratic Estimates}  
\label{tab:MRP-GA}
\begin{tabular}{rlrccc}
  \hline \\ \vspace{0.5em}
Year & CVAP & MRP & 90\% CI & Two-Party Share & MRPe \\ 
  \hline
2000 & 5,456,066 & 0.39 & 0.305-0.478 & 0.44 & 0.05 \\ 
  2004 & 5,796,787 & 0.40 & 0.293-0.513 & 0.42 & 0.02 \\ 
  2008 & 6,358,125 & 0.48 & 0.420-0.557 & 0.47 & 0.01 \\ 
  2012 & 6,650,237 & 0.48 & 0.439-0.541 & 0.46 & 0.02 \\ 
  2016 & 7,003,492 & 0.47 & 0.342-0.483 & 0.47 & 0.00 \\ 
  2020 & 7,482,258 & 0.51 & 0.486-0.642 & 0.50 & 0.00 \\ 
   \hline \vspace{0.5em}
     &          &       &       &     &        MAE: 0.016   \\
\end{tabular}
\end{threeparttable}
\end{table}

\subsubsection{Voting Patterns by Race/Ethnicity}

The analysis proceeds by inferring partisan preferences by race/ethnicity for White, Black, Hispanic, and Other categories. It bears repeating that previous research has not produced strong empirical evidence of state-level voting by race/ethnicity for this period. ANES sample sizes are inadequate, and in some years do not including a single respondent from either Hispanic or Other race/ethnic categories (Table \ref{table:ANES-respondents-race}). 


Academics have relied chiefly on the results of exit polls, which were conducted for the National Election Pool media consortium for every presidential contest except 2012 -- no small omission -- as reported in table \ref{tab:exits-both}. State-level exit polls are often considered unreliable \citep{wright_misreports_1990, barreto_controversies_2006, morin_report_2005}, however they provide a basic benchmark for MRP estimates of race/ethnic vote preferences. Starting in 2008, the Cooperative Election Study conducted large-scale surveys, including reasonable sample sizes in Georgia, that can be used to further validate MRP estimates for the later years of the period of interest. Like the exit polls, they track with the model-based MRP estimates.

\begin{table}[ht]
\centering
\begin{threeparttable}
\caption{MRP vs. Exit Polls}  
\label{tab:exits-both}
\begin{tabular}{rcccrr}
  \hline \\ \vspace{0.5em}
Year & White Voters & 90\% CI & Exit Polls & Diff \vspace{0.5em} \\ 
  \hline
2000 & 0.23 & 0.15-0.32 & 0.26 & 0.03  \\ 
  2004 & 0.26 & 0.15-0.36 & 0.23 & 0.03  \\ 
  2008 & 0.25 & 0.18-0.40 & 0.23 & 0.02   \\ 
    2012 & 0.26 & 0.20-0.33 & 0.20 & 0.03   \\ 
2016 & 0.20 & 0.14-0.29 & 0.21 & 0.01  \\ 
  2020 & 0.32 & 0.29-0.48 & 0.30 & 0.02 \\
   \hline
\end{tabular}
{\footnotesize Source: Voter News Service, YouGov, Edison Research}
\end{threeparttable}
\end{table}

In figure \ref{fig:ga-race} the race/ethnic breakdown for the MRP estimates is displayed. The probability of voting Democratic is low among White voters and remains fairly stable at this level, with a notable uptick in Democratic support at the end. Black voters are seen at peak support in 2008, when Barack Obama won election to the presidency. Obama's 47-percent vote share surprised many observers, coming hot on the heels of Bush's 17-point victory of the Democrats in 2004, and following the GOP's takeover of the majority of both state legislative chambers, securing a near-monopoly on state government power. It showed the liberal multiracial coalitions being built by the Democratic party could put up a fight in Georgia, particularly if minority voters were organized. According to the Census Bureau's CPS-based estimates of Georgia voter turnout, 68 percent of the Black CVAP turned out to vote in 2008, higher than the 64-percent turnout rate of the non-Hispanic Whites CVAP. Estimated preferences of Hispanic voters only appear in 2012 and 2020, and are substantially higher than the MRP estimates at 67 percent and 58 percent, respectively. The odds produced by poststratification are shown in printed table \ref{tab:mrp-race-est}, with both the estimates and 90 percent credible intervals.


\begin{figure}[ht]
\centering
\includegraphics[width=\textwidth]{UTSAthesisPackage/pics/GArace2000.png}
\caption{Voting preference by four race/ethnic groups, 2000-2020}
\label{fig:ga-race}
\end{figure}


One vital characteristic of a critical election is that enough voters defect so as to change the outcome of an election as part of the broader shift of party power implied by realignment. This does not equate an upset election with realignment, since swing voters with weak or no partisan ties regularly produce surprises at the ballot box. However, registering such a flip is an important contribution to the case for partisan realignment. The relative impact of shifting preferences among White voters. A straightforward estimate of racial vote switching can be conducted by simulating the election results using MRP estimated party voting and official turnout demographics.

Georgia records race and ethnic information for residents when they register to vote, one of only a handful of U.S. states to do so. The Secretary of State's Elections Division publishes Turnout Demographic files that report registration and voting activity by race and ethnicity, sex, and age at geographies from the state as a whole down to the voting precinct. The reporting is not necessarily consistent: in some years counties are the smallest identifiable unit, for example. In addition, the Turnout files frequently do not match the official results. At the precinct level the variance can be quite high and the discrepancies between specific precincts perplexing, aggravated by such policies as coding precincts differently in the Turnout and Results reports and failing to disclose records of registered voters purged from the rolls. To accommodate the precinct count mismatches, participation by race/ethnicity is first calculated from the Turnout files, and then these proportions are used to distribute the official tallies. If turnout data reports that half of the 112 voters in a given election were White and half were non-White, but there are only 100 votes reported in the official results, 50 ballots are assigned to White voters, and 50 to non-Whites. 

\begin{table}[ht]
\begin{center}
\begin{threeparttable}
\caption{Share of Federal Election Ballots by Race, 2000-2020} 
\label{tab:vottes-race-pct}
\begin{tabular}{rrrrr}
  \hline \\ \vspace{0.5em}
Year & White & Black & Hispanic & Other \vspace{0.5em} \\ 
  \hline
2000 & 75.1\% & 23.2\% & --- & --- \\ 
  2004 & 71.4\% & 25.4\% & 0.0\% & 1.1\% \\ 
  2008 & 64.1\% & 30.1\% & 1.1\% & 1.9\% \\ 
  2012 & 61.4\% & 29.9\% & 1.3\% & 1.9\% \\ 
  2016 & 60.8\% & 27.7\% & 2.1\% & 2.7\% \\ 
  2020 & 58.2\% & 27.3\% & 3.0\% & 4.4\% \\ 
   \hline 
\end{tabular}
\end{threeparttable}
\end{center}
\end{table}


Another issue to overcome is that Georgia permits voters to decline to identify their race/ethnicity, and categorizes these demurrals as `unknown.' Unknown and Hispanic were both introduced as categories in 2004. Because that category does not have a match in the ANES data, voters of unknown race/ethnicity are not modeled alongside White, Black, Hispanic, and Other. Using precinct-level data, the partisan preferences of voters with unknown race/ethnicity are estimated using ecological inference via the eiCompare package by creating `known'` and `unknown' voter column to regress against Democratic and Republican votes \citep{collingwood_eicompare_2020}. 


\begin{table}[ht]
\begin{center}
\begin{threeparttable}
\caption{Ecological regression estimates of Democratic support}
\label{tab:unknowns}
\begin{tabular}{rrr}
  \hline \\ \vspace{0.5em}
Year & Unknown  & All Voters \vspace{0.5em}\\ 
  \hline
  2000 & 49\% & 43\% \\ 
  2004 & 37\% & 41\% \\ 
2008 & 48\% & 47\% \\ 
  2012 & 47\% & 45\% \\ 
  2016 & 52\% & 45\% \\ 
  2020 & 55\% & 49\%\vspace{0.5em}\\ 
   \hline
\end{tabular}
\end{threeparttable}
\end{center}
\end{table}

Next, the estimated Democratic voting preference for each race/ethnic group is applied to the number of adjusted number of ballots reported by the Georgia Secretary of State in their respective category. The number of ballots cast for each party for each election year by each race/ethnic group is calculated. so as to measure the cycle-to-cycle evolution of partisan balance. The estimated series of votes by race/ethnic group begins with the year 2000, such that ballot changes in percent and count are recorded for 2004 and forward. The year 2000 is an appropriate base year. Aside from being the millennial election, it was the watershed between Georgia's transitional elections of the 1990s, which saw important Democratic electoral victories and the decade of consolidated Republican control in the `00s. The 1996 election would be less useful as a base year because of the three-way competition involving independent candidate Ross Perot. Republican Bob Dole won by a 1-percent victory margin, defeating Bill Clinton, who had prevailed in 1992 thanks to the spoiler effect of Perot. The 2004 election, on the other hand, could set an unreasonably high of comparison for the Republican party, with little room to go but down. Republican candidate Bush swept 58 percent of the two-party vote share, receiving more than half a million ballots than his Democratic opponent. That election also produced the Republican ascent to trifecta. 


\begin{table}[ht]
\centering
\caption{Poststratified Estimates by Race/Ethnicity}
\label{tab:mrp-race-est}
\begin{threeparttable}
\begin{tabular}{rcccc}
   \hline \\ \vspace{0.5em}
Year & White & Black & Hispanic & Other \vspace{0.5em} \\
  \hline
2000 & 0.23 (0.152-0.324) & 0.808 (0.717-0.882) & 0.285 (0.179-0.419) & 0.327 (0.205-0.463) \\ 
  2004 & 0.244 (0.145-0.362) & 0.784 (0.666-0.878) & 0.416 (0.262-0.582) & 0.273 (0.134-0.438) \\ 
  2008 & 0.253 (0.173-0.348) & 0.974 (0.965-0.993) & 0.539 (0.416-0.669) & 0.584 (0.447-0.742) \\ 
  2012 & 0.258 (0.204-0.329) & 0.943 (0.930-0.968) & 0.557 (0.481-0.651) & 0.405 (0.335-0.508) \\ 
  2016 & 0.202 (0.139-0.294) & 0.801 (0.749-0.839) & 0.518 (0.411-0.626) & 0.346 (0.254-0.459) \\ 
  2020 & 0.34 (0.287-0.480) & 0.818 (0.865-0.934) & 0.522 (0.519-0.728) & 0.446 (0.385-0.591) \vspace{0.5em} \\ 
   \hline
\end{tabular}
\end{threeparttable}
\end{table}


\subsubsection{Results: MRP Estimates Applied to Race/Ethnic Votes}

Vote-switching in table \ref{tab:mid-white} reflects partisan preference changes that suggest behavior characteristic of partisan dealignment in 2016 and partisan realignment in 2020. When voters are alienated from the electoral process at a differential rate between parties, the macropartisan change is produced via the demobilation thesis. In 2016, total ballots cast by White voters were flat with 2012. Votes cast for Clinton declined 1.2 percent from 2012, while those cast for Trump fell by 2.3 percent versus Romney's tally. The White citizen voting-aged population rose 1.6 percent, so there was a relative decline in turnout. At first blush, the outcome doesn't seem particularly dramatic, given the small margins. It bears mention that while Georgia voters exhibited less enthusiasm about the vote, the U.S. overall saw an increased turnout rate. What does stand out is that the volume of ballots cast by White voters for third-party candidates rose 127.7 percent, a pattern that generally accompanies realignment. 

The `net' column in table \ref{tab:mid-white} is the net increase to the Democratic vote tally, or the difference in the year-over-year change in both the Democratic and Republican vote counts. Thus if the Democratic ticket gains 10 supporters between elections, but the Republican party gains 5 supporters, the net is 5 votes; Republican increases are expressed as negative figures. The differential in major party votes equated to a 34,850 net benefit for the Democratic party, as Trump turned off more White voters than did Clinton. Non-white voters, meanwhile, cast more Democratic and Republican ballots alike in 2016 versus 2012, by 4.0 percent and 39.6 percent, respectively, producing a net Democratic decline of 20,667. It is worth noting that the Republican gain compares with a low base, and in terms of count, the increase to Democratic and Republican tallies were 43,404 and 63,971, respectively. The net gain for the GOP column was 20,667. Total minority turnout climbed by 11.1 percent, on par with a CVAP increase of 11.9 percent. The net effect of changes in race/ethnic partisan preference across the board summed to 93,720, which narrowed Trump's victory margin to 211,141. 




\begin{table}[ht]
\centering
\begin{threeparttable}
\caption{Estimated Party Votes by Race/Eth, Mean Scenario}  
\label{tab:mid-white}
\begin{tabular}{rrrrrrr}
  \hline \\ \vspace{0.5em} 
Year & White Dem & White Rep & Net Change & Minority Dem  & Minority Rep  & Net Change \vspace{0.5em} \\ 
  \hline   \\
2000 & 580,978 & 1,311,746 &   -- & 535,250 & 107,973 &   -- \\ 
  2004 & 619,663 & 1,680,782 & -330,351 & 732,486 & 212,924 & 92,285 \\ 
  2008 & 616,066 & 1,915,247 & -238,062 & 1,175,540 & 74,357 & 581,621 \\ 
  2012 & 584,075 & 1,834,208 & 49,048 & 1,094,197 & 130,328 & -137,314 \\ 
  2016 & 577,135 & 1,792,418 & 34,850 & 1,137,501 & 194,299 & -20,667 \\ 
  2020 & 988,150 & 1,899,559 & 303,874 & 1,292,099 & 407,346 & -58,449   \vspace{0.5em} \\ 
   \hline

\end{tabular}
\end{threeparttable}

\end{table}




In 2020, White voters turned out in large numbers, casting 19.2-percent more ballots than in 2016, with a large proportion of this surge of enthusiasm accruing to the Democratic ticket. Biden won nearly 1,000,000 votes from the White electorate, an increase of 411,015, or 71.2 percent, versus White support for Clinton. White Republican turnout rose 6.0 percent, producing 107,141 additional votes for Trump in his re-election bid. The net difference amounted to 303,874 additional votes for the Democratic ticket, considering only White voters. Minorities also voted in larger numbers than in 2016, adding 13.6 percent and 136.0 percent to the Democratic and Republican counts, respectively. On balance, the net effect raised 58,449 Republican votes from minorities. Voters who did not declare their race/ethnicity shuffled 22,505 votes toward the Republican column versus 2016. 

In sum, the net gains White voters contributed to the Democratic purse less the net gain for Trump equated to 222,920 additional Democratic votes. Given that Trump lost to Biden by a slender margin of 11,779 votes, and that minority and unknown race/ethnicity voters shifted support to the Republican ticket, the change in White voter partisan preferences was the decisive factor in the Democratic upset of 2020. The gain in White support was in fact larger than Trump's winning margin of 211,141 in 2016. The changes in voting preferences by race/ethnicity in the past three cycles showed steady net Democratic gains among White voters, albeit it modestly until 2020 when the shift was substantial -- enough to swing the result. Minority voters, meanwhile, improved Trump's chances of winning in the same period. This contradicts the narrative that minority voter mobilization drove Georgia's electoral college votes to flip. 

\subsubsection{Simulation: Low White Democratic Support Scenario}


However, the estimates of White voter preference change are presented with some uncertainty. The credible interval contains 80 percent of the most frequently simulated outcomes, with the MRP estimate representing the mean of all draws. To take the most conservative, credible approach to the analysis, the estimate can be fixed at the 10-percent lower credible interval bound. In order to recalculate the race/ethnic partisan preferences on this basis, the lower-bound estimate replaces the mean estimate as the multiplier which is applied to the reported number of ballots of each race/ethnic group. For clarity, only the probable Democratic preferences of White voter are lowered for the low-scenario simulation. Third-party votes by race/ethnicity are held constant, as are the total number of ballots per race/ethnic group. After recalculating White votes for the Democratic and Republican candidates, the balance of the total is used to adjust the minority vote tally. Results are shown in \ref{tab:lo-white}.        




\begin{table}[ht]
\centering
\begin{threeparttable}
\caption{Simulated Election Results, Low White Scenario}  
\label{tab:lo-white}
\begin{tabular}{rrrrrrr}
  \hline \\ \vspace{0.5em} 
Year & White Dem & White Rep & Net Change & Minority Dem  & Minority Rep  & Net Change \vspace{0.5em} \\ 
  \hline   \\
  2000 & 448,618 & 1,270,390 &   -- & 667,611 & 149,329 &  -- \\ 
  2004 & 506,546 & 1,623,611 & -295,293 & 845,678 & 270,188 & 57,208 \\ 
  2008 & 481,160 & 1,896,147 & -297,922 & 1,314,149 & 90,425 & 648,234 \\ 
  2012 & 506,618 & 1,816,397 & 105,208 & 1,176,435 & 143,762 & -191,051 \\ 
  2016 & 473,926 & 1,723,843 & 59,862 & 1,250,104 & 246,446 & -29,015 \\ 
  2020 & 878,160 & 1,956,163 & 171,914 & 1,413,453 & 338,428 & 71,367     \vspace{0.5em} \\ 
   \hline

\end{tabular}
\end{threeparttable}

\end{table}



Even at the lower bound of credible estimates, White voters are seen shifting ballots to the Democratic party. In the low-scenario, Trump clings onto Georgia's electoral votes with a razor-thin margin of 2,978 votes in 2020. There is a net gain in White votes for the Democratic party in 2012, 2016, and 2020. As in the mean scenario, voters in general reduce support for major parties in 2016, and of a lesser magnitude from the Democratic tally than from the Republican, 3.4 percent versus 6.5 percent, respectively. The net benefit to the Democratic party is 59,862 White votes. In the minority column, the low White scenario contributes 29,015 additional votes to Trump's haul. Thus, neither White nor minority voters would have produced a Clinton victory under the assumption; Trump wins by 246,249 votes rather than 211,141. Next, in the Trump-Biden contest of 2020, there are 109,990 fewer expected votes from the White constituency under this scenario, reducing the net gain to 171,914. While still a substantial boost, it is no longer the only component driving the change in the partisan balance. Minority votes increased by 13.1 percent and 37.3 percent for the Democratic and Republican candidates, respectively, in 2020 versus 2016. On a net basis, Biden sees a bounce of 71,367 minority votes in this scenario. Despite the Democratic ticket gaining minority support between the 2016 and 2020 elections in the low scenario, instead of registering a net loss as occurs in the mean scenario, Trump would have beaten Biden. The Democratic candidate required the additional White support estimated in the mean scenario in order to defeat his rival. The findings here are less unambiguous than in the mean scenario, but they are still significant: Even using the most conservative credible estimate, White vote-switching still creates a large net increase in Democratic votes.   

\subsubsection{Simulation: Constant Registered Turnout Rates}


Another factor to be considered in the assessment of voter behavior is turnout, which varies by such factors as geography, race/ethnicity, age, educational attainment, income, marital status, and party identification. Actual turnout by race/ethnicity is known, as both the number of registered voters and ballot counts by race/ethnicity are reported by the Secretary of State. But both the demographic and partisan compositions of the turnout population could be varying across time. For example, liberal Whites who were habitual nonvoters prior to 2016-2020 may have been motivated by displeasure with Trump's controversial policies to participate, producing the change in estimated Democratic voting odds of Whites. Thus, the 2020 upset could have resulted from a mobilization effect, not conversion, driving the partisan realignment. Or, as another example, turnout may have fallen among White Republicans, but not among White Democrats, indicating that dealignment, not conversion or mobilization, led to rebalancing the distribution of partisanship by race. 



\begin{table}[ht]
\centering
\begin{threeparttable}
\caption{Registration Change by Race/Ethnicity}  
\label{tab:chg-reg-race}
\begin{tabular}{rrrrrrr}
  \hline \\ \vspace{0.5em} 
 & Year & White & Black & Hispanic & Other & Unknown  \vspace{0.5em} \\ 
  \hline
 & 1996 &--  &--  & -- & -- &  \\ 
   & 2000 & -0.01 & 0.05 & -- & 0.42 & -- \\ 
   & 2004 & 0.04 & 0.18 &--  & -0.03 & -- \\ 
   & 2008 & 0.12 & 0.35 & 1.43 & 0.46 & 1.90 \\ 
   & 2012 & -0.03 & 0.03 & 0.25 & 0.10 & 0.89 \\ 
   & 2016 & 0.18 & 0.27 & 0.80 & 0.64 & 0.56 \\ 
   & 2020 & 0.08 & 0.12 & 0.66 & 0.68 & 0.23\vspace{0.5em} \\ 
   \hline
\end{tabular}
{\footnotesize Note: Net change between elections in number of registered voters.}\\
{\footnotesize Source: Georgia Secretary of State}
\end{threeparttable}
\end{table}



In 2016 and 2020, the population of registered voters posted remarkable gains across race/ethnic groups. With the exception of an 8-percent growth of White registrants in 2020, the rolls grew by double-digit across the board. Those with lower bases of comparison, namely Hispanic and Other voters, showed the most growth. However, the larger denominator may have contributed to a lower turnout rate in relative terms if aggressive registration drives reached a considerable number of low-propensity voters who did not reach the polls. The Covid-19 pandemic may have exerted a negative effect, but even so both 2016 and 2020 had lower-than-average turnout rates for registered voters in the period of interest. Moreover, research has shown that while voter registration drives tend to boost low-SES voters more than high-SES voters, they generate roughly the same gains in overall turnout, since low-SES voters turnout at lower rates \citep{nickerson_voter_2015}. This further muddies the analysis, since the SES profiles of White and non-White populations differ substantially in Georgia. Turnout of registered White voters fell from a high point in 2008 to a rate in 2020 that was second only to the year 2000 in terms of poor turnout. Black voters exhibited a similar drop, though the 2012 election was a high-water mark for Black electoral participation. Turnout rates fell to the their lowest points in the period of interest for all other race/ethnic voter groups. Notably, the differential between White and Black voting incidence was the largest of any year in the 2000 to 2020 time frame, with White turnout 11.6 points higher than Black turnout.



\begin{table}[ht]
\centering
\begin{threeparttable}
\caption{Turnout Rates by Race/Ethnicity}  
\label{tab:ave-to-rates}
\centering
\begin{tabular}{rrrrrrrr}
  \hline \\ \vspace{0.5em} 
 & Year & White & Black & Hispanic & Other & Unknown & Total \vspace{0.5em} \\
  \hline
 & 2004& 0.80 & 0.72 & 0.60 & 0.66 & 0.53 & 0.69\\ 
 & 2008 & 0.77 & 0.76 & 0.60 & 0.62 & 0.60 & 0.77\\ 
  & 2012 & 0.76 & 0.73 & 0.56 & 0.58 & 0.60 & 0.76  \\ 
   & 2016& 0.68 & 0.56 & 0.54 & 0.52 & 0.50 & 0.73\\ 
   & 2020& 0.73 & 0.60 & 0.55 & 0.61 & 0.51 & 0.62\\ 
  & \textbf{Ave.} & \textbf{0.75} & \textbf{0.67} & \textbf{0.57} & \textbf{0.60} & \textbf{0.55} & \textbf{0.66} \vspace{0.5em}\\ 
   \hline
\end{tabular}
{\footnotesize Source: Georgia Secretary of State}
\end{threeparttable}
\end{table}



To control for turnout variation, one approach is to maintain participation rates steady across elections. To achieve this, the turnout rates of registered voters by race/ethnicity are averaged from 2004 to 2020, and then applied to the party vote preferences estimates. Turnout rates are obtained by dividing the official tally by the number of registered voters by race/ethnicity. The year 2000 is excluded because Hispanic and Unknown statistics were first published in 2004. The modified turnout rates are applied to the official count of registered voters to obtain new vote tallies. Following the same procedure as before, the vote share of each group is calculated and the official vote tally is redistributed by race/ethnicity. As evidenced in table \ref{tab:constant-reg-to}, the new counts can vary substantially when turnout assumptions are held constant, with nearly 100,000 additional White votes in 2008 marking the group's largest variation, and 131,474 added ballots from Black voters in 2020. In the two key elections, 2016 and 2020, the White vote count declines by 9,125 and increases by 14,380, respectively, versus observed counts, while minorities in aggregate gain in both elections, by 965,630 and 122,943 ballots. New ballot totals are presented in table \ref{tab:constant-reg-to}.




\begin{table}[ht]
\small
\centering
\begin{threeparttable}
\caption{Ballots by Race/Ethnicity, Actual and Estimated by Averaged Turnout}  
\label{tab:constant-reg-to}
\begin{tabular}{rrrrrr|rrrrr}
 \hline \\ \vspace{0.5em}
   &  &  &Observed  &  &  &  &  & Estimated &  & \\    \hline  
  Year & White & Black & Hisp. & Other & Unk. & White & Black & Hisp. & Other & Unk.  \\
  \hline
 2000 & 1.94 & 0.60 & -- &.04 & --&1.94 & 0.60 & -- & .04 & --   \\ 
   2004 & 2.35& 0.84 & 0.02 & 0.05 & 0.03 & 2.36 & 0.84 & 0.19 & 0.05 & 0.04 \\ 
   2008 & 2.52 & 1.18& 0.04 & 0.07 & 0.11 & 2.61 & 1.12 & 0.45 & 0.08 & 0.11 \\ 
   2012 & 2.39& 1.17 & 0.05 & 0.07 & 0.21 & 2.48 & 1.13 & 0.55 & 0.08 & 0.21 \\ 
   2016 & 2.49& 1.13 & 0.09 & 0.11 & 0.28 & 2.48 & 1.23 & 0.08 & 0.11 & 0.27 \\ 
   2020 & 2.91 & 1.37 & 0.15 & 0.22 & 0.35 & 2.92 & 1.50 & 0.15 & 0.22 & 0.37\vspace{0.5em}\\ 
   \hline
\end{tabular}
{\footnotesize Note: Figures expressed represent millions of ballots}\\
{\footnotesize Source: Georgia Secretary of State and author's estimates}
\end{threeparttable}
\end{table}



Under this scenario, the Democratic ticket gains less White voter support than in the mean scenario, adding 18,566 versus 34,870 in 2016, and 232,882 versus 303,874 in 2020. Applying average 2004-2020 turnout rates to the base of registered minority voters alters the partisan balance in terms of outcome. The mean scenario estimates that rising minority support produced a net increase of 20,667 and 58,449 ballots to Trump's column in 2016 and 2020, respectively. However, when the results are simulated with those same preferences and averaged turnout rates, minority constituencies produce a net gain of 125,341 Republican votes in 2016 and a net gain of 99,638 Democratic votes in 2020. Including a slight change in the Unknown race/ethnicity partisan tallies, Georgia would have handed a resounding victory to Biden in 2020 under this scenario, with a margin of 147,041 over Trump. In the previous electoral cycles, the averaged turnout scenario does not reverse Republican wins, though it does reduce the GOP's two-party votes share versus the observed results. A prima facie reading of the simulated outcomes is simply that, in line with conventional wisdom, higher turnout would have generally benefited the Democratic party (but see \citet{martinez_effects_2005}, and that this effect is stronger for minority Democratic voters than for White counterparts. The takeaway from the average-turnout simulation is that if turnout levels remained constant across time 1) White voters still would have defected to the Democratic ticket in hefty numbers, such that they were the largest factor in a Democratic ticket win in 2020, and 2) there may be a more nuanced narrative underlying the small Republican shift of minority voters estimated in the mean scenario. 



\begin{table}[ht]
\centering
\begin{threeparttable}
\caption{Simulated Election Results, Averaged Registered Turnout Scenario}  
\label{tab:mid-reg-to}
\begin{tabular}{lrrrrrr}
  \hline \\ \vspace{0.5em} 
Year & White Dem & White Rep & Net Change & Minority Dem  & Minority Rep  & Net Change \vspace{0.5em} \\ 
  \hline   \\
  2000* & 580,978 & 1,311,746 &   -- & 535,250 & 107,973 &   -- \\ 
  2004 & 576,807 & 1,777,442 &   -- & 683,635 & 224,760 &   -- \\ 
  2008 & 661,292 & 1,906,275 & -4,4347 & 1,163,407 & 71,855 & 632,677 \\ 
  2012 & 639,813 & 1,769,711 & 115,084 & 1,132,248 & 124,764 & -84,067 \\ 
  2016 & 500,647 & 1,611,678 & 18,866 & 1,066,136 & 183,993 & -125,341 \\ 
  2020 & 994,293 & 1,872,442 & 232,882 & 1,397,556 & 415,775 & 99,638     \vspace{0.5em} \\ 
   \hline
\end{tabular}
{\footnotesize *Year 2000 is not adjusted for turnout}
\end{threeparttable}
\end{table}



\subsubsection{Simulation: Constant CVAP Turnout Rates}

However, this exercise does not control for changes in the electorate stemming from population changes rather than potential registration organizing. The growth of the registered voter base far outstripped change in the CVAP. Georgia's citizen adult population increased by 13.6 percent between the 2010 ACS 1-Year and the 2019 ACS estimates (1-year 2020 ACS estimates were not released due to Covid-related quality concerns), whereas the registered voter base grew 51.8 percent between 2010 and 2020. The White CVAP increased 4.2 percent in that time frame, while all other race/ethnic categories posted double-digit growth, as displayed in \ref{tab:cvap-chg-to}. For example, the Hispanic CVAP expanded 77.4 percent, but the number of registered Hispanic voters essentially tripled that pace at 236-percent growth. To reduce the influence of voter registration activism on the simulation results, another turnout scenario is developed using turnout estimates of the citizen voter-aged population.   


\begin{table}[ht]
\centering
\begin{threeparttable}
\caption{CVAP v. Registered Voter Growth 2010-2019}  
\label{tab:cvap-chg-to}
\begin{tabular}{llrrr}
  \hline \\ \vspace{0.5em} 
 Year&   Race/Ethnicity & CVAP & CVAP Chg.& Reg. Chg.\\ 
   \hline
2010 & White &4,191,292 &  --&--\\ 
   & Black & 2,063,161 & -- &--\\ 
    &Hispanic & 215,295 & -- &--\\ 
    &Other & 201,630  & -- &--\\ 
  2020 & White &4,367,617 & 4\% & 30\%\\ 
    &Black & 2,493,514 & 21\% &    56\% \\ 
    &Hispanic & 381,959 & 77\%  &   236\% \\ 
    &Other &338,747 & 68\% &   211\% \\ 
   \hline
\end{tabular}
{\footnotesize Source: ACS 1-Year, 2010 \& 2019; Georgia Secretary of State Voter Turnout Files}
\end{threeparttable}
\end{table}



This potentially reduces the racially differential impact of voter registration campaigns, but introduces several new challenges. Firstly, while voter registration numbers are officially recorded, Census estimates for the denominator are published with a margin of error, increasing the uncertainty of the simulation. There are two primary Census sources from which CVAP turnout denominators are generally developed: the American Community Survey and the Current Population Survey Voter Supplement. The 5-year ACS is of high quality and is used to produce the Bureau's own Citizen Voting Age Population by Race and Ethnicity Special Tabulation. For a large geography such as the State of Georgia, 1-year ACS data are the same high quality and more appropriate, since federal elections are spaced four years apart. However, the 2020 ACS 1-year survey was not released officially after quality reviews revealed substantial nonresponse bias related to the pandemic, with specific problems related to estimates of educational attainment, marital status, and citizenship \citep{shin_assessment_2021}. Hence, users may use either the 2019 1-year or 2016-2020 5-year microdata to construct CVAP estimates for the 2020 election. The ACS does not record any information regarding voting behavior. 

While the ACS represents the gold standard for ACS estimates, the CPS, likewise a high quality Census survey with a large sample, conducts a supplemental survey every other November measuring voting behavior, the universe for which is the civilian, noninstitutionalized citizen adult population living in the United States. The first item asks whether or not the respondent voted in the most recent federal election, with a coding response listed for `not in universe,' and the next item inquires if the individual is registered to vote. The follow-up includes a response for those who are not eligible, which helps narrow the citizen adult voting population to the voting eligible population. Population controls for the CPS are tied to the 2010 Decennial Census, which are updated monthly via the Population Estimates program, and may be expected to deviate slightly from the true population toward the end of each decade, before being tuned to, in this case, the 2020 Census. The CPS voter report is known to contain overreporting bias, as do virtually all electoral surveys, and unlike many contemporary large nongovernmental election studies, responses to the CPS regarding voting or registration are not validated against official state registration and voting records. The CPS also excludes a fraction of the electorate, such as institutionalized or overseas absentee voters. Hence there is a degree of error in the CPS sample, the presence of unvalidated overreporting bias, and some coverage gaps. When conducting MRP poststratification, it is desirable that the survey sampling frame match the Census table, and the most common electoral surveys target the the CVAP, not the voter eligible population (VEP). For the purposes of the turnout simulation however, it is desirable to approximate the true turnout rate. \citet{hur_coding_2013} suggest identifying the VEP by eliminating CPS respondents who do not have either a `voted' or `did not vote' response in the supplemental questionnaire, and then downweighting each respondent by dividing each by the state-level VEP turnout estimates from the U.S. Election Project \citep{mcdonald_united_2021}. 


% 

% \begin{table}[ht]
% \centering
% \begin{threeparttable}
% \caption{Estimated 2004-2020 Turnout by Race/Ethnicty}  
% \label{tab:cvap-to}
% \begin{tabular}{rlrrrc}
%   \hline \\ \vspace{0.5em} 
% Year & Race/Ethnicity & Ballots & CVAP & Turnout & vs. Average \vspace{0.5em}  \\ 
%   \hline  
% 2004 & White & 2,354,384 & 3,896,788 & 0.60 & 0.61 \\ 
%   & Black & 837,794 & 1,623,400 & 0.52 & 0.56 \\ 
%   & Hispanic & 18,316 & 117,529 & 0.16 & 0.29 \\ 
%   & Other & 53,536 & 159,070 & 0.34 & 0.42 \\ 
%   & Unknown & 34,768 & 65,061 & 0.53 & 0.53 \\ 
%   2008 & White & 2,515,946 & 4,136,037 & 0.61 & 0.61 \\ 
%   & Black & 1,179,533 & 1,860,605 & 0.63 & 0.56 \\ 
%   & Hispanic & 43,607 & 171,078 & 0.25 & 0.29 \\ 
%   & Other & 72,471 & 190,405 & 0.38 & 0.42 \\ 
%   & Unknown & 112,929 & 188,722 & 0.60 & 0.53 \\ 
%   2012 & White & 2,392,800 & 4,145,338 & 0.58 & 0.61 \\ 
%   & Black & 1,165,100 & 2,003,205 & 0.58 & 0.56 \\ 
%   & Hispanic & 51,688 & 266,083 & 0.19 & 0.29 \\ 
%   & Other & 74,935 & 235,611 & 0.32 & 0.42 \\ 
%   & Unknown & 213,316 & 357,605 & 0.60 & 0.53 \\ 
%   2016 & White & 2,487,580 & 4,201,724 & 0.59 & 0.61 \\ 
%   & Black & 1,132,720 & 2,203,126 & 0.51 & 0.56 \\ 
%   & Hispanic & 87,075 & 298,559 & 0.29 & 0.29 \\ 
%   & Other & 109,735 & 300,083 & 0.37 & 0.42 \\ 
%   & Unknown & 275,263 & 558,090 & 0.49 & 0.53 \\ 
%   2020 & White & 2,910,014 & 4,337,883 & 0.67 & 0.61 \\ 
%   & Black & 1,366,586 & 2,390,746 & 0.57 & 0.56 \\ 
%   & Hispanic & 151,299 & 372,783 & 0.41 & 0.29 \\ 
%   & Other & 217,255 & 374,783 & 0.58 & 0.42 \\ 
%   & Unknown & 352,562 & 688,975 & 0.51 & 0.53\vspace{0.5em}  \\ 
%   \hline
% \end{tabular}
% {\footnotesize Source: CPS, ACS; Georgia Secretary of State Voter Turnout Files}

% \end{threeparttable}
% \end{table}



\begin{table}[H]
\centering
\begin{threeparttable}
\caption{Estimated 2004-2020 Turnout by Race/Ethnicty}  
\label{tab:cvap-to}
\begin{tabular}{rlrrrc}
  \hline \\ \vspace{0.5em} 
Year & Race/Ethnicity & Ballots & CVAP & Turnout &  Average \vspace{0.5em}  \\ 
  \hline  
2004 & White & 2,354,384 & 3,896,784 & 0.60 & 0.61 \\ 
   & Black & 837,794 & 1,623,378 & 0.52 & 0.56 \\ 
   & Hispanic & 18,316 & 117,529 & 0.16 & 0.26 \\ 
   & Other & 53,536 & 159,070 & 0.34 & 0.40 \\ 
   & Unknown & 34,768 & 65,061 & 0.53 & 0.55 \\ 
  2008 & White & 2,515,945 & 4,136,029 & 0.61 & 0.61 \\ 
   & Black & 1,179,532 & 1,860,601 & 0.63 & 0.56 \\ 
   & Hispanic & 43,607 & 171,078 & 0.25 & 0.26 \\ 
   & Other & 72,471 & 190,405 & 0.38 & 0.40 \\ 
   & Unknown & 112,929 & 188,722 & 0.60 & 0.55 \\ 
  2012 & White & 2,392,800 & 4,145,340 & 0.58 & 0.61 \\ 
   & Black & 1,165,100 & 2,003,195 & 0.58 & 0.56 \\ 
   & Hispanic & 51,688 & 266,083 & 0.19 & 0.26 \\ 
   & Other & 74,935 & 235,611 & 0.32 & 0.40 \\ 
   & Unknown & 213,315 & 357,605 & 0.60 & 0.55 \\ 
  2016 & White & 2,487,580 & 4,201,727 & 0.59 & 0.61 \\ 
   & Black & 1,132,719 & 2,203,113 & 0.51 & 0.56 \\ 
   & Hispanic & 87,075 & 298,559 & 0.29 & 0.26 \\ 
   & Other & 109,735 & 300,083 & 0.37 & 0.40 \\ 
   & Unknown & 275,263 & 558,090 & 0.49 & 0.55 \\ 
  2020 & White & 2,910,014 & 4,337,890 & 0.67 & 0.61 \\ 
   & Black & 1,366,585 & 2,390,752 & 0.57 & 0.56 \\ 
   & Hispanic & 151,299 & 372,783 & 0.41 & 0.26 \\ 
   & Other & 217,255 & 374,783 & 0.58 & 0.40 \\ 
   & Unknown & 352,562 & 688,975 & 0.51 & 0.55\vspace{0.5em}  \\ 
   \hline
\end{tabular}
{\footnotesize Source: CPS, ACS; Georgia Secretary of State Voter Turnout Files}

\end{threeparttable}
\end{table}



This study once again takes advantage of the Georgia Turnout Demographics voter files to pursue a more flexible and accurate method. In a first step, the ACS is used to construct a poststratification table of the US with the demographic characteristics of race and ethnicity. Population count estimates are generated with the Survey package in R, specifying the survey's complex design elements of weights, cluster, and strata \citep{lumley_survey_2020}. For each race/ethnic group, the design object is subset to the population of noninstitutionalized adult citizens aged 18 or more and residing in a U.S. state. A second design object is created from the CPS November survey. Following \citet{hur_coding_2013}, the sample is restricted to voters who replied `yes' or `no' to the primary voting question. From this object, a poststratification table is constructed with person weights controlled to the ACS subpopulation totals for each race/ethnicity. Next, respondents who reported as having voted, coded `2' in the CPS voting data, are controlled to the officially reported number of votes by race/ethnic group. This purges any overreporting bias from the results. The denominator is controlled to the ACS 1-year surveys, the gold standard for CVAP estimates. As the Census Bureau is not officially releasing an authorized 1-year 2020 file, the 2016-2020 5-year survey is utilized for 2020. The numerator is tied to the official Georgia Turnout Demographics file. Some minor discrepancies remain due to the absence of an ACS control for 'Unknown'. The denominator for this category substitutes the number of registered voters as a proxy. 



\begin{table}[ht]
\centering
\begin{threeparttable}
\caption{Simulated Election Results, Averaged CVAP Turnout Scenario}
\label{tab:mid-cvap-to}
\begin{tabular}{lrrrrrr}
  \hline \\ \vspace{0.5em} 
Year & White Dem & White Rep & Net Chg. & Minority Dem  & Minority Rep  & Net Chg.\\ 
  \hline
  2000* & 580,978 & 1,311,746 &   -- & 535,250 & 107,973 &   -- \\ 
   2004 & 574,469 & 1,768,795 & -1,194,325 & 679,065 & 223,315 &  455,749 \\ 
  2008 & 636,562 & 1,853,455 & -22,566 & 1,214,651 & 75,285 & 683,615 \\ 
  2012 & 617,926 & 1,736,874 & 97,943 & 1,157,613 & 126,785 & -108,537 \\ 
  2016 & 502,491 & 1,898,416 & -276,976 & 990,381 & 300,516 & -340,961 \\ 
  2020 & 989,404 & 1,884,624 & 500,706 & 1,293,741 & 415,156 & 188,718\vspace{0.5em} \\ 
   \hline
\end{tabular}
{\footnotesize *Year 2000 is not adjusted for turnout}
\end{threeparttable}
\end{table}

When CVAP-based turnout rates are averaged and applied to estimated CVAP numbers in each election, the outcomes do not change: including Unknown votes (not displayed in Table \ref{tab:mid-cvap-to}), the Republican ticket wins in 2004, 2008, 2012, and 2016, and loses in 2020. In 2020, Biden defeats Trump by 25,008 votes, more than double the true outcome, and a bigger ask to make when calling the Georgia Secretary of State post-election in search of ``lost votes,'' but still a slim victory. Notably, the aggregate ballot count is 4.09 million versus 4.05 million in 2016, and 5.00 million versus 4.99 million in 2020, in the CVAP and mean scenarios, respectively, indicating the methodology for estimating turnout is sound. Both White and non-White voters would have shifted strongly toward Trump in 2016, only to turn around in 2020 with approximately one million net additional Democratic votes.    

The variation in White voting estimates is fairly modest, not unexpected given that both the base population and turnout propensities are relatively stable across elections. Though the magnitude of change in the net Democratic margin differs, the net tendency is the same for all years except 2016. White voters begin the period of Republican trifecta with strong support for the GOP, and in 2012 begin to swing to the Democratic side. Using an average turnout rate magnifies the swinginess of White voters compared with the mean scenario, but nevertheless in the gross White ballot counts are essentially the same. Minority voters display increasingly Republican partisan voting patterns across the period, as in the mean scenario, except for 2020, when they swing heavily toward the Democratic candidate. Smoothing out the pronounced variability in turnout rates over an electorate that grew significantly in size magnifies the swings even more than is seen in the White voters during this simulation. In the final analysis, once the volatility of the CVAP simulation is set aside, the overall picture remains: White voters defected in droves in 2020, and in large enough numbers to award Georgia's electoral votes to Joe Biden.              

\subsection{Extending the Poststratification Table}

As a final step, the poststratification table of census variables is extended to include two variables from ANES: partisan ID and campaign contact. The addition of noncensus variables to the MRP methodology was suggested by \citet{kastellec_polarizing_2015}. In this study their proposal is extended in a novel usage.  A multilevel model is fit to predict partisanship based on the same variables used to predict voter preference. The mean of posterior draws is then poststratified on the original census data to predict partisanship for each cell, such that the MRP estimate reports what proportion of Whites, Blacks, Hispanics, and Other race/ethnic groups identify as Democrats, Republicans, or Independents. This is extended again using the same procedure to estimate the proportion of the CVAP that reported in the ANES as having been contacted by one of the major party campaigns. By poststratifying the posterior draws of the second model onto the extended poststratification table, estimates are generated for proportions of the electorate contacted by a campaign by partisan ID and by race. To illustrate, sample rows from the twice-extended poststratification table are shown in Table \ref{tab:twice-ps}. 

\begin{table}[H]
\centering
\caption{Extended Poststratification Table (sample rows)}
\label{tab:twice-ps}
\begin{threeparttable}
\begin{tabular}{rlllllccr}
  \hline \\ \vspace{0.5em} 
Year & Race & Sex & Age & Marital & Education & Party ID & Contacted & CVAP \\ 
  \hline
2020 & White & Male & 55-64 & Married & High School & Rep & No  &34,181 \\ 
  2020 & Black & Female & 18-24 & Single & Some College & Dem & Yes& 34,111\\ 
  2020 & White & Male & 45-54 & Married & BA/BS & Rep & Yes &33,404\\ 
  2020 & Black & Female & 18-24 & Single & Some College & Dem & No&33,157 \\ 
  2020 & White & Male & 45-54 & Married & BA/BS & Rep & No& 33,029 \\ 
  2020 & White & Female & 35-44 & Married & BA/BS & Dem & Yes&  32,249\\ 
  \hline
\end{tabular}
\end{threeparttable}
\end{table}

Two separate multilevel models are fit on the ANES 2000-2020 surveys, voter turnout and preference. These are poststratified on the extended poststratification table. First, the proportion of voters versus nonvoters are estimated. For each race/ethnic group, the share of ballots by party ID and contact are obtained by multiplying the turnout rate by CVAP for each cell. Estimated turnout rates are substantially higher using the ANES survey and a standard self-reported dichotomous response variable than those for those using the methodology developed for Table \ref{tab:cvap-to}. As such, these shares are redistributed using the Georgia Secretary of State's Turnout Demographic counts to control for overreporting bias in the survey. Although overreporting (and other survey-based biases) likely vary among race/ethnic groups and by partisanship, the aim is not to simulate elections at this stage, but rather develop evidence of the factors driving the vote-switching. 

\begin{table}[H]
\centering
\caption{Calculating Predicted Democratic Votes (2020 sample rows)}
\label{tab:extended-MRP}
\begin{threeparttable}
\begin{tabular}{rlllrrr}
  \hline \\ \vspace{0.5em} 
Rear & Race/Eth & Party ID& Contacted & Turnout MRP & Dem. Odds MRP  \\ 
  \hline
2020 & White & Dem & No & 0.86 & 0.95  \\ 
  2020 & White & Dem & Yes & 0.92 & 0.96  \\ 
  2020 & White & Ind & No & 0.56 & 0.54 \\ 
  2020 & White & Ind & Yes & 0.69 & 0.57  \\ 
  2020 & White & Rep & No & 0.81 & 0.09  \\ 
  2020 & White & Rep & Yes & 0.89 & 0.10  \\
   \hline
\end{tabular}
\end{threeparttable}
\end{table}

Several important points emerge from this step regarding campaign outreach. First, White voters and non-White voters reported being contacted by the Republican or Democratic party with the same frequency in 2016 and 2020. In the 2016 cycle, the rates ranged from 17 percent to 20 percent of the electorate reporting contact, and in 2020 from 44 percent to 46 percent, and in both elections the differences among race/ethnic groups were not significant. Prior to 2016, White voters tended to report contact at higher rates than Black voters, with the exception of 2008. The 2020 election had the highest rate of campaign outreach of any year in the 2000-2020 period; 2016 had the lowest. Second, in all years except 2008, White Democratic partisans were contacted at higher rates than minority Democratic voters, though by marginal differences. In 2016 and 2020, 21 percent and 50 percent of all Democratic voters were contacted, versus an average of 20 percent and 47 percent of minority Democratic voters, respectively. In 2008, 55 percent of Black Democrats reported being contacted versus 49 percent of White Democrats. Third, Republican partisans report being contacted by a major party campaign at lower rates than Democratic partisans, but the rate does not vary among White and minority voters. For example, in the 2016 cycle, the rates ranged from 17 to 20 percent, matching the average for all partisans, while in the 2020 the range went from 41 percent to 42 percent --- lower than the average, but also without statistically significant differences among race/ethnic groups. Fourth, independent voters report less contact than either Democratic or Republican partisans, but again there is little variation among race/ethnic groups. Campaign outreach unsurprisingly boosted turnout, and the effects were much more dramatic for independent voters.

\begin{table}[H]
\centering
\caption{Democratic Votes by Race \& Campaign Contact}
\label{tab:extended-demvotescount}
\begin{threeparttable}
\begin{tabular}{rrrrrrr}
  \hline \\ \vspace{0.5em} 
Year& White Contacted & White Not Contacted & Min Contacted & Min Not Contacted \\ 
  \hline
2000  & 303,087 & 301,712 & 174,005 & 337,426 \\ 
  2004  & 509,454 & 197,837 & 378,654 & 280,209 \\ 
  2008  & 456,052 & 278,182 & 680,165 & 429,724 \\ 
  2016  & 175,071 & 492,252 & 288,137 & 922,505 \\ 
  2020  & 557,173 & 538,838 & 682,600 & 695,019 \\ 
   \hline
\end{tabular}
\end{threeparttable}
\end{table}

In the 2020 election, a roughly equal number of ballots came from voters who reported being contacted by a campaign, and this was true of both White and minority voters. Oddly enough, in 2016 having contact with a political campaign appeared to depress the Democratic vote. This was also true for both White and minority voters. In fact, because of the slump in 2016 among contacted voters, it is difficult to draw strong conclusions from Table \ref{tab:extended-demvotescount}. However, voters who were contacted and who were not contacted both contributed roughly an equal number of ballots to the Biden tally, which does not correspond to the narrative that strong grassroots organizing propelled the Democratic ticket to victory. Republican voters (results not displayed), both White and Minority, were less likely to report having been contacted in 2020, but this was true for all 2000-2020 electoral cycles. On balance, while the activation of nonvoters cannot be ruled out as a factor in the Democratic party's turn of fortunes, the evidence is inconclusive.       


\begin{table}[H]
\centering
\caption{Democratic Votes by Race \& Party ID, Annual \% Chg.}
\label{tab:extended-demvotes}
\begin{threeparttable}
\begin{tabular}{rrrr|rrr}
  \hline \\ 
   &    &White  &   &   & Min  &   \\ \vspace{0.5em} 
 &   Dem ID & Ind ID &  Rep ID &  Dem ID &  Ind ID &  Rep ID \\ 
  \hline
 2004 & -2.0 & 529.1 & 0.4 & 25.4 & 121.5 & 73.5 \\ 
  2008 & 9.7 & -66.6 & 123.3 & 64.4 & 95.1 & 249.9 \\ 
  2016 & -0.9 & -50.3 & -33.1 & 14.3 & -43.5 & -48.8 \\ 
  2020 & 53.7 & 386.1 & 49.4 & 8.5 & 130.3 & 118.4 \\ 
   \hline
\end{tabular}
\end{threeparttable}
\end{table}

In terms of partisanship and vote-switching, twice as many Republicans voted for Biden than did Democrats for Trump. An estimated 11 percent of all White votes for Biden were cast by Republican partisans. On the other hand, only 3 percent of White votes for Trump came from Democratic partisans. The net effect of party-switching among White voters amounted to nearly 55,000 additional votes for Biden --- enough to have cost Trump the election. White and minority voters who identify as independent swung strongly toward the Democratic ticket in 2020. The independent category excludes leaners, following standard practice. Democratic leaning independents are counted as Democrats, and the same for Republican leaners. More significantly, however, White and minority Republican partisans defected their votes to Biden by 49 and 118 percent, respectively, compared with 2016. White Republican defectors alone increased their votes by 39,000 over 2016. Also noteworthy, more White Republican partisans voted for Biden than White independents, 119,076 versus 108,306. Independent partisans voting Democrat could signal conversion, but the shift of Republican partisans represents much stronger evidence of a realignment process. According to the extended poststratification table, 36 percent of Georgia's CVAP identified as Republican in 2020 --- down from 42 percent in 2016 and 45 percent in 2012. And 54 percent of the White electorate identifies with the GOP, down from 62 percent and 66 percent in 2016 and 2012, respectively. Hence, the results do indicate that conversion was a factor in the 2020 upset, and the changing composition of partisan ID strongly indicates that realignment may have been taking place during the entire Trump presidency.     


\section{Discussion and Conclusion}

All three hypotheses are supported by poststratified estimates. The first is only partially supported in that a large movement occurred in 2020, but not in 2016. However, an usual pattern emerged that could qualify 2016 as a marker of a dealignment phase preceding realignment in 2020. The magnitude of the growth in White Democratic ballots from 2016 to 2020 vastly exceeds any major-party swings in the period of interest, including the Republican gains of the 2004 election of George Bush, a watershed Republican victory. Given this, it is unsurprising that the second hypothesis is confirmed: The movement of White voters to the Democratic column produced Biden's victory. Moreover, it overcame a minority voters shift toward the GOP to so, validating the third hypothesis.  

The implications of these findings are without doubt significant. The Republican party recognizes the ineluctable arithmetic of demographic change in the American electorate. A national committee report on the 2012 loss of candidate Mitt Romney declared, ``The nation's demographic changes add to the urgency of recognizing how precarious our position has become.'' Two years later, when Trump launched his campaign with a fiery speech laced with anti-immigrant and racist invective, South Carolina Republican Senator Lindsey Graham called him, `a race baiting, xenophobic religious bigot,' and a `wrecking ball for the Republican party.' But Trump's strategy was informed by voices within the party which had been arguing that the GOP would reap more return on investment by doubling down on White voters, boosting their turnout, suppressing that of minorities, and courting ``downscale, blue-collar White voters'' \citep{trende_lost_1994, trende_demographics_2015}. Both the fierce loyalty of his voter base and the capitulation of the party elite to the business magnate's grab for the party's reins suggest that out-group racial resentment has has found purchase in the GOP. This could be due to the increased importance of primary elections amid a monotonic decline in competitive electoral districts in the U.S. Primary election candidates compete for votes of strong partisans, who are more conservative (or liberal in Democratic primaries) than voters in a general election. Though the GOP has refined its colorblind racism strategy over time, the subtleties of that paradigm compete against Trumpesque messaging, which may be proving more popular, or at least more accessible, with the Republican base. When it comes to politics, audible whistles resonate louder than dog whistles.       

Yet these results clearly indicate that as a significant constituency of White voters in a conservative southern state clearly rejected the Republican party's current trajectory, and they did so at a time when the GOP appeared willing to alienate minority voters in exchange for energizing racial conservatism among the White electorate. In Georgia, and probably elsewhere, this tactic cost the party the election --- and quite possibly future elections as well, given the stark possibility --- supported by the evidence here --- that the secular drift of White voters toward the Democratic party received a critical push toward realignment and not simply a passing rejection of candidate Trump. 

Reviewing federal elections in the 2000 to 2020 period, White voters began the period with a strong swing toward the Republican party. The Democratic column picked up nearly 100,000 White voter ballots in 2004, but this largely reflected a rebound in turnout -- voter turnout was a mere 46 percent in 2000, jumping to 56 percent in 2004 \citep{mcdonald_united_2021}. By 2008, the GOP had outpaced Democratic gains in presidential elections by half-a-million votes. Thereafter, the tide turned, though initially the change was small and difficult to discern amid a sharp drop in turnout. In 2012, a low-turnout election, the Democratic and Republican tickets saw declines of 4.6 percent and 4.3 percent versus 2008, and a small net benefit accrued to the Democratic ticket. In 2016, the major parties either saw a negligible gain or a decline, whereas third-party votes nearly tripled. Again, this is suggestive of a pattern of partisan dealignment. Normally, partisan identification colors how voters receive information, a mechanism that links micro- and macropartisanship \citep{carsey_changing_2006}. Simply put, `conversion' is unlikely to happen to average partisans, even if the party they affiliate with adopts unfavorable policies, simply because voters are prone to adjust their views until policies seem favorable -- moving toward the party rather than the party moving toward the voter. However, in extraordinary moments, public sentiment may be shaken en masse, decoupling that link. Once the rose-tinted glasses are removed, conversion becomes more probable \citep{darmofal_dynamics_2010}. 

In the case of Georgia, 2016 has the hallmarks of a dealignment election: though turnout rates and ballot counts were very similar in 2016 compared with 2012, White voters cast fewer Republican ballots and voted for third party and independent candidates at the highest rate since 1996. Subsequently, in 2020 White voters supported Biden with an unprecedented surge of 66 percent compared with Hillary Clinton's tally in 2016. This indicates that the realignment that appears to be underway is driven by conversion far more than by cohort progression or mobilization. It remains a possibility that 2020 was an exceptional swing prompted by a polarizing candidate and the highly unusual circumstances created by the Covid-19 pandemic. However, the elements proposed by V.O. Key and refined in the literature since then were certaintly present: a dealigned electorate, a time of national crisis, social and political polarization, and a voting change of remarkable magnitude. Should this trend hold, Georgia's macropartisan balance will have been tipped away from the Republicans, at least in presidential voting, but quite possibly for state-level offices as well. Although the Republican party's dominance in the Peach State was not unassailable, it might be said that Georgia was now theirs to lose.  


% \subsubsection{Modeling Poststratification Output}

% In order to test hypotheses regarding racial threat, models are fit on the results of the poststratification procedure. The response variable is the continuous measure of $voted\_dem$, the probability of each demographic group voting Democrat. Although the CCES data have sample sizes large enough to estimate the probability of voting Democratic ticket without the poststratification step, the cumulative time series survey only has data on partisan voting preference from 2008 to 2020. Whereas the initial hierarchical logistic models used for MRP estimation were run separately for each electoral cycle, here the poststratified results are combined to produce a time series that can take advantage of both ANES and CCES covariates.  


% MRP has been used primarily to generalize estimates for population cells, rather than for direct hypothesis testing. As discussed, the models are restricted in what variables they can accept, governed by matching variables in the census data. The extension of the poststratification table performed in the previous chapter is not practical for most applications, especially if the hypothesis testing requires a number of noncensus variables. The approach calls for deep interactions of random effects, producing saturated output that is not readily interpreted. This paper adopts a novel strategy. First, the hierarchical regressions are fit on the survey data and diagnostics are examined. The estimates are then poststratified by combinations of geographic and demographic strata and plotted. Finally, in a `post' poststratification step, the MRP estimates are joined back to the survey by characteristic. For example, if White voters with a college education are estimated to have a 50-percent probability of supporting the Democratic ticket at the polls in 2012, respondents with those characteristics are coded with a response variable with the value 0.50. The predictor is thus continuous, and theoretically important independent variables relevant to the hypotheses are modeled on the outcome. The results are presented with interpretation.         


% The motivating idea behind this chapter's hypotheses is that the 2016 and 2020 presidential contests, with the racially divisive figure of Donald Trump on the ticket, have at least met the preconditions to qualify as critical elections as well. Conceivably, the 2024 election will represent the denouement of a trilogy. It is further suggested that the coalition to emerge will prominently include former Southern White Republican voters with higher levels of educational attainment, a refinement and contextualization of the so-called diploma divide of the U.S. population \citep{highton_cultural_2020, harris_america_2018, harris_not_2020}. 

% In the long term, the strategy is doomed. On the one hand, demographic processes are glacial: change comes slowly, metered in generational time, but also steady, unceasing, and futile to resist. Politics, on the other, has a long and proven history of producing swift and unanticipated disruption to the status quo. Following the nonlinear corollary, trafficking in conflict to aggravate a racial threat response among some White voters may have the unintended consequence of prompting others to abandon ship --- and seek safe harbor in the Democratic party instead. 
